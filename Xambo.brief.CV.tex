	%------------------------------------
% Dario Taraborelli
% Typesetting your academic CV in LaTeX
%
% URL: http://nitens.org/taraborelli/cvtex
% DISCLAIMER: This template is provided for free and without any guarantee 
% that it will correctly compile on your system if you have a non-standard  
% configuration.
% Some rights reserved: http://creativecommons.org/licenses/by-sa/3.0/
%------------------------------------

%!TEX TS-program = xelatex
%!TEX encoding = UTF-8 Unicode

\documentclass[10pt, a4paper]{article}
\usepackage{fontspec} 

% DOCUMENT LAYOUT
\usepackage{geometry} 
\geometry{a4paper, textwidth=7in, textheight=10.1in, marginparsep=8pt, marginparwidth=.7in}%original: marginparsep=7pt, marginparwidth=.6in
\setlength\parindent{0in}

% FONTS
\usepackage[usenames,dvipsnames]{xcolor}
\usepackage{xunicode}
\usepackage{xltxtra}
\defaultfontfeatures{Mapping=tex-text}
%\setromanfont [Ligatures={Common}, Numbers={OldStyle}, Variant=01]{Linux Libertine O}
%\setmonofont[Scale=0.8]{Monaco}
%%% modified by Karol Kozioł for ShareLaTeX use
\setmainfont[
  Ligatures={Common}, Numbers={OldStyle}, Variant=01,
  BoldFont=LinLibertine_RB.otf,
  ItalicFont=LinLibertine_RI.otf,
  BoldItalicFont=LinLibertine_RBI.otf
]{LinLibertine_R.otf}
\setmonofont[Scale=0.8]{DejaVuSansMono.ttf}

% ---- CUSTOM COMMANDS
\chardef\&="E050
\newcommand{\html}[1]{\href{#1}{\scriptsize\textsc{[html]}}}
\newcommand{\pdf}[1]{\href{#1}{\scriptsize\textsc{[pdf]}}}
\newcommand{\doi}[1]{\href{#1}{\scriptsize\textsc{[doi]}}}
% ---- MARGIN YEARS
\usepackage{marginnote}
\newcommand{\amper{}}{\chardef\amper="E0BD }
\newcommand{\years}[1]{\marginnote{\scriptsize #1}}
\renewcommand*{\raggedrightmarginnote}{} %\original: raggedleftmarginnote
\setlength{\marginparsep}{8pt}%original: 7pt
\reversemarginpar

% HEADINGS
\usepackage{sectsty} 
\usepackage[normalem]{ulem} 
\sectionfont{\mdseries\upshape\Large}
\subsectionfont{\mdseries\scshape\normalsize} 
\subsubsectionfont{\mdseries\upshape\large} 

% PDF SETUP
% ---- FILL IN HERE THE DOC TITLE AND AUTHOR
\usepackage[%dvipdfm, 
bookmarks, colorlinks, breaklinks, 
% ---- FILL IN HERE THE TITLE AND AUTHOR
	pdftitle={Anna Xambó Sedó - vita},
	pdfauthor={Anna Xambó Sedó},
	pdfproducer={http://nitens.org/taraborelli/cvtex}
]{hyperref}  
\hypersetup{linkcolor=blue,citecolor=blue,filecolor=black,urlcolor=MidnightBlue,linkcolor=MidnightBlue} 

% AVOID WIDOW LINES
\usepackage[all]{nowidow}

%HEADER & FOOTER
\usepackage{lastpage}
\usepackage{fancyhdr}
\pagestyle{fancy}
\renewcommand{\headrulewidth}{0pt}
% Footer
%\lfoot{Anna Xambó, PhD}
%\cfoot{Curriculum Vitae}%removes pagination at the center of the footer
%\rfoot{\thepage\ of \pageref{LastPage}}

% CUSTOM
\usepackage{microtype}
%\hyphenpenalty 10000
%\exhyphenpenalty 1000
%\widowpenalty 10000
%\clubpenalty 10000
%\interfootnotelinepenalty=10000


% DOCUMENT
\begin{document}

\pagenumbering{gobble} % this suppresses page numbering

{\textbf{Anna Xambó Sedó}}\\ 
Senior Lecturer in Sound and Music Computing, Centre for Digital Music, QMUL -- \href{https://annaxambo.me/}{annaxambo.me}\\

{\textbf{Education}}\\
\textsc{PhD}, Music computing \& HCI, The Open University (OU), 2015.\\
MSc, Information, Communication and Audiovisual Media Technologies, Universitat Pompeu Fabra (UPF), 2008.\\
BA, MA, Social and Cultural Anthropology, Universitat de Barcelona, 1996.\\

{\textbf{Employment}}\\
Senior Lecturer, QMUL, 2024--present. I have been senior lecturer, De Montfort University (DMU) (2020--2023); associate professor, Norwegian University of Science and Technology (NTNU) (2018--19) where I also was Music Communication \& Technology (MCT) master programme council leader and MCT programme study leader at NTNU; visiting lecturer and postdoc at Queen Mary University of London (2017--19); and postdoc at Georgia Tech (2015--17). Also, I have been co-founder and chair of the organization Women Nordic Music Technology (WoNoMute) (2018--19) and co-founder and co-chair of the organization Women in Music Tech at Georgia Tech (2016--17). From 2013--14, I was a research fellow at the London Knowledge Lab, University College London Institute of Education.
I have been working for 10 years (2000--10) as interaction designer, web designer and web developer with strong background on Internet technologies. I have been employee in companies (2000--10) as well as co-founder of my own interactive media studio (2004--10).\\

{\textbf{Grants/Awards/Accolades}} from QMUL ERIC Fund (2025); UKRI AHRC (2023--25); DMU Living in a Digital Society Spotlight (2022); DMU Future Research Leaders Programme (2021-22); UKRI EPSRC/HDI Network (2020--21); NTNU Starting Grant (2018--20); Sonic Arts Research Center (2018); NCWIT (2017); NSF AISL (2016); WiMIR (2016); OU (2010--13, 2010); Fund. Caixa de Sabadell (2003--04); and Generalitat de Catalunya (1998--99, 2001--02).\\

{\textbf{Peer-Reviewed Journal Publications}} in Journal of New Music Research (2025); Organised Sound (2023); MDPI (2021); Computer Music Journal (2019); Journal of Audio Engineering Society (2018, 2020); Interacting with Computers (2017); International Journal of Social Research Methodology (2017); Qualitative Research (2017); ACM Transactions on Computer-Human Interaction (2013); and Information Processing \& Management (2013).\\

{\textbf{Conference and Festival Participation}} at New Interfaces for Musical Expression (NIME) (2008, 2011, 2014, 2017--22); Web Audio Conference (WAC) (2017--22); Audio Mostly (2010, 2017--18); Tangible, Embedded, and Embodied Interaction Conference (TEI) (2011, 2017--18); International Conference on Computational Creativity (2017); International Conference on Live Coding (ICLC) (2019, 2025); International Society for Music Information Retrieval Conference (2016); Sound and Music Computing (SMC) Conference (2010, 2012, 2016); International Conference of Live Interfaces (ICLI) (2016, 2020); Research in Equity and Sustained Participation in Engineering, Computing, and Technology (2015); CHI (2014); International Workshop on Content-Based Multimedia Indexing (2013); British Computer Society Human-Computer Interaction (2011--12); and International Computer Music Conference (2011).\\

{\textbf{Invited Keynote Talks}} at ADC (2023); WAC (2021); SMC (2020); Women in Sound Women on Sound (2016).\\

{\textbf{Performances}} at +RAIN (2023); Cafe OTO (2022); British Science Festival (2022); BEAST (2022); EarTaxi (2021); Jefferson Park EXP (2021); Network Music Festival (2020); Eulerroom Equinox (2020); PACE/DMU (2020--21); ICLC (2019); Dokkhuset (2018); Sonic Arts Research Center (2018); Inter/sections (2018); WAC (2017--22); Cube Fest (2018); NIME (2017--18, 2021--22); TEI  (2018); Audio Mostly (2017); Root Signals Festival (2017); Freedonia (2017); Women in Sound Women on Sound (2016); Phonos (2013); Niu (2012); Mostra Sonora i Visual (2006); Minima Festival (2004); and Sonar Festival (2002).\\

{\textbf{Conference Organizer}} for WAC (2019, general co-chair; 2017, session chair; 2016, music/artworks co-chair); NIME (2019, paper co-chair, session chair); ICLI (2020, conference local committee member); Computational Creativity (2017, local committee member).\\

{\textbf{Interviews/Press Coverage}} in PSF (2023); The Wire (2022); OnCurating.org (2020); Ballade.no (2019); Girls Geek Dinner Trondheim (2019); ACM Interactions (2018); Netlabel Interview Project (2018); and Wallifornia Music Tech (2018).\\

{\textbf{Research Visits}} at Filmuniversität Babelsberg KONRAD WOLF (2018--2019); University of Strathclyde (2011--12); UPF (2011); and OU (2010).\\

{\textbf{Blog Coordinator}} at Sensing the Forest (2023--); MIRLCAuto (2020--21); WoNoMute (2018--2019); AudioCommons (2017--18); Women in Music Tech (2016--17); Methodological Innovation in Digital Arts and Social Sciences (2013--14); postWIMP (2010--11); and streetTypes (2006--09).

%\vspace{1cm}
%\vfill{}
%\hrulefill

\begin{center}
{\scriptsize  Anna Xambó Sedó •\- Brief Curriculum Vitae •\- Last updated: \today\- •\- %original: Last updated: \today\- •\- 
% ---- PLEASE LEAVE THIS BACKLINK FOR ATTRIBUTION AS PER CC-LICENSE
Typeset in \href{http://nitens.org/taraborelli/cvtex}{
%\fontspec{Times New Roman}
\XeTeX } % ---- FILL IN THE FULL URL TO YOUR CV HERE
•\- 
\href{http://github.com/axambo/CV}{http://github.com/axambo/CV}}
\end{center}

\end{document}