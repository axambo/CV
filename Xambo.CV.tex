	%------------------------------------
% Dario Taraborelli
% Typesetting your academic CV in LaTeX
%
% URL: http://nitens.org/taraborelli/cvtex
% DISCLAIMER: This template is provided for free and without any guarantee 
% that it will correctly compile on your system if you have a non-standard  
% configuration.
% Some rights reserved: http://creativecommons.org/licenses/by-sa/3.0/
%------------------------------------

%!TEX TS-program = xelatex
%!TEX encoding = UTF-8 Unicode

\documentclass[10pt, a4paper]{article}
\usepackage{fontspec} 

% DOCUMENT LAYOUT
\usepackage{geometry} 
\geometry{a4paper, textwidth=5.5in, textheight=8.5in, marginparsep=8pt, marginparwidth=.7in}%original: marginparsep=7pt, marginparwidth=.6in
\setlength\parindent{0in}

% FONTS
\usepackage[usenames,dvipsnames]{xcolor}
\usepackage{xunicode}
\usepackage{xltxtra}
\defaultfontfeatures{Mapping=tex-text}
%\setromanfont [Ligatures={Common}, Numbers={OldStyle}, Variant=01]{Linux Libertine O}
%\setmonofont[Scale=0.8]{Monaco}
%%% modified by Karol Kozioł for ShareLaTeX use
\setmainfont[
  Ligatures={Common}, Numbers={OldStyle}, Variant=01,
  BoldFont=LinLibertine_RB.otf,
  ItalicFont=LinLibertine_RI.otf,
  BoldItalicFont=LinLibertine_RBI.otf
]{LinLibertine_R.otf}
\setmonofont[Scale=0.8]{DejaVuSansMono.ttf}

% ---- CUSTOM COMMANDS
\chardef\&="E050
\newcommand{\html}[1]{\href{#1}{\scriptsize\textsc{[html]}}}
\newcommand{\pdf}[1]{\href{#1}{\scriptsize\textsc{[pdf]}}}
\newcommand{\doi}[1]{\href{#1}{\scriptsize\textsc{[doi]}}}
% ---- MARGIN YEARS
\usepackage{marginnote}
\newcommand{\amper{}}{\chardef\amper="E0BD }
\newcommand{\years}[1]{\marginnote{\scriptsize #1}}
\renewcommand*{\raggedrightmarginnote}{} %\original: raggedleftmarginnote
\setlength{\marginparsep}{8pt}%original: 7pt
\reversemarginpar

% HEADINGS
\usepackage{sectsty} 
\usepackage[normalem]{ulem} 
\sectionfont{\mdseries\upshape\Large}
\subsectionfont{\mdseries\scshape\normalsize} 
\subsubsectionfont{\mdseries\upshape\large} 

% PDF SETUP
% ---- FILL IN HERE THE DOC TITLE AND AUTHOR
\usepackage[%dvipdfm, 
bookmarks, colorlinks, breaklinks, 
% ---- FILL IN HERE THE TITLE AND AUTHOR
	pdftitle={Anna Xambó - vita},
	pdfauthor={Anna Xambó},
	pdfproducer={http://nitens.org/taraborelli/cvtex}
]{hyperref}  
\hypersetup{linkcolor=blue,citecolor=blue,filecolor=black,urlcolor=MidnightBlue,linkcolor=MidnightBlue} 

% AVOID WIDOW LINES
\usepackage[all]{nowidow}

%HEADER & FOOTER
\usepackage{lastpage}
\usepackage{fancyhdr}
\pagestyle{fancy}
\renewcommand{\headrulewidth}{0pt}
\lfoot{Anna Xambó, PhD}
\cfoot{Curriculum Vitae}%removes pagination at the center of the footer
\rfoot{\thepage\ of \pageref{LastPage}}

% CUSTOM
\usepackage{microtype}
\usepackage{eurosym}
\usepackage{xcolor}
%\hyphenpenalty 10000
%\exhyphenpenalty 1000
%\widowpenalty 10000
%\clubpenalty 10000
%\interfootnotelinepenalty=10000


% DOCUMENT
\begin{document}
{\Huge Anna Xambó}\\[0.1cm]
\textsc{BA, MA, MSc, PhD}\\[0.9cm]
\emph{Senior Lecturer in Music and Audio Technology}\\
Leicester Media School\\
Faculty of Computing, Engineering and Media\\
De Montfort University (DMU)\\
Clephan Building \\
De Montfort University\\
Leicester LE1 9BH United Kingdom\\[.2cm]
%\textsc{phone}: \texttt{(+44) (0) 737-879-6645}\\
\textsc{email}: \href{mailto:anna.xambo@dmu.ac.uk}{anna.xambo@dmu.ac.uk}\\
\textsc{webpage}: \href{http://annaxambo.me/}{annaxambo.me}

%%\hrule
\section*{Current Academic Roles}
---\emph{Senior Lecturer in Music and Audio Technology}, Leicester Media School, Faculty of Computing, Engineering and Media, DMU, Leicester, UK.\\
---\emph{Member of Music, Technology and Innovation - Institute for Sonic Creativity (MTI${^2}$)}, Leicester Media School, Faculty of Computing, Engineering and Media, DMU, Leicester, UK.\\
---\emph{BSc Digital Music Technology Programme Leader}, Leicester Media School, Faculty of Computing, Engineering and Media, DMU, Leicester, UK.\\
---\emph{ Associate Fellow}, Higher Education Academy, United Kingdom.\\
---\emph{Women in NIME (WiNIME) Officer}, International Conference of New Interfaces for Musical Expression.
%%\hrule
\section*{Areas of Interest}
%Design of Digital Musical Instruments (DMIs) • Real-time Interactive Systems for Music Performance • Human-Computer Interaction • Interaction Design • Tangible, Physical \& Social Computing • Computer-Supported Collaborative, Participatory \& Improvisation Music • Live Coding • Real-time Music Information Retrieval and Machine Learning • Multichannel Spatialization • Generative \& Algorithmic Music • Immersive Sound Experiences • Women in Music Tech • Arts \& Social Sciences Research Methods • STEAM Education • Data Visualization • Creative Programming
Sound and Music Computing • New Interfaces for Musical Expression • Human-Computer Interaction • Tangible Interaction • Computer-Supported Collaborative Music • Live Coding • Music Information Retrieval • Machine Learning • Generative \& Algorithmic Music • Spatial Audio • Women in Music Tech • STEAM Education • Data Visualization • Data Science • Creative Computing • Network Music • Web Audio • Audio Electronics • Audio DSP • Creative AI • Sonic Arts

%\hrule
\section*{Education}
\noindent
\years{2015}\textsc{PhD}, The Open University (OU), UK \& \textsc{Dra.}, Universitat Pompeu Fabra (UPF), Spain.\\
Major: Music computing \& HCI.\\ 
Dissertation: \emph{Tabletop Tangible Interfaces for Music Performance: Design and Evaluation}.\\
\years{2008}\textsc{MSc} in Information, Communication and Audiovisual Media Technologies, UPF, Spain.\\
Major: Music computing \& HCI.\\ 
Dissertation: \emph{Interfaces for Sketching Musical Compositions}.\\
\years{1999}\textsc{Master} in Video, Animation and Multimedia Design, Media Art Institute Fak d'Art, Spain.\\
\years{1996}\textsc{BA, MA} in Social and Cultural Anthropology, Universitat de Barcelona (UB), Spain.

%\hrule
\section*{Dissertation}
\noindent
\years{Title}\textbf{Xambó, A.} (2015). \emph{Tabletop Tangible Interfaces for Music Performance: Design and Evaluation}.\\
\years{Advisors}Dr Robin Laney, Mr Chris Dobbyn and Prof Sergi Jordà.\\
\years{Examiners}Prof Eduardo Reck Miranda and Dr Janet van der Linden.\\
\years{Website}\href{http://oro.open.ac.uk/42473/}{http://oro.open.ac.uk/42473/}

%\hrule
\section*{Music Education}
%\noindent
\subsection*{Classical Training}
\years{1983--1987}\textsc{Piano}, Conservatori Superior de Música del Liceu, Barcelona.\\
\years{1982--1988}\textsc{Music Theory \& Solfege}, Conservatori Superior de Música del Liceu, Barcelona.
\noindent
\subsection*{Workshops}
\years{2018}\textsc{Spatial Audio Workshop} by Eric Lyon. Virginia Tech, Blacksburg, Virginia, USA.\\
\years{2014}\textsc{Taller composición acusmática} (Acousmatic Composition Workshop) by Beatriz Ferreyra. Barcelona.\\
\years{2012}\textsc{Síntesi no estàndard: tècniques, estètiques, extensions} (Non-Standard Synthesis: Techniques, Aesthetics, Extensions) by Luc Döbereiner. Barcelona.\\
\years{2009}\textsc{Taller construeix el teu propi sintetitzador} (Build Your Own Synthesizer Workshop) by Tom Bugs. Barcelona.\\
\years{2008}\textsc{SMC Summer School} by Xavier Serra, Marc Leman, Benjamin Knapp, and the Casa Paganini - InfoMus Lab. Genoa, Italy.\\
\years{2006}\textsc{El món com a instrument} (The world as an instrument) by Francisco López. Barcelona.\\
\years{1998}\textsc{Improvització mètode Cobra} (Cobra Improvisation Method) by Orquestra del Caos. Barcelona.

%\hrule
\section*{Teaching Education}
\noindent
\years{2020}\textsc{Research Supervision} (with certificate). Instructors: Sally Ruane (module 1), Deborah Cartmell (module 2), Meera Warrier (module 3). DMU. Leicester, UK.\\
\years{2018--2019}\textsc{Learning and Teaching in Higher Education} (with certificate). Instructors: Emma Kennedy (module 1), Alison Gilmour (module 2), Maren Thom (module 2). QMUL. London.\\
\years{2018}\textsc{Women into Leadership}. Instructor: Lorraine Smith. QMUL. London.\\
\years{2017}\textsc{Communication Skills for Teaching for International Faculty, Postdocs, and Visiting Scholars}. Instructor: Katherine Samford. Georgia Institute of Technology (Georgia Tech). Atlanta, GA, USA.

%%\hrule
\section*{Employment}
\noindent
\years{01/2020--present}\textsc{Senior Lecturer in Music and Audio Technology}. Leicester Media School, Faculty of Computing, Engineering and Media, DMU. \\
\years{08/2018--12/2019}\textsc{Associate Professor in Music Technology}. Department of Music, NTNU. \\
\years{08/2018--01/2019}\textsc{Visiting Lecturer}. C4DM, School of Electronic Engineering and Computer Science (EECS), Queen Mary University of London (QMUL). \\
\years{10/2017--07/2018}\textsc{Postdoctoral Research Assistant}. C4DM, School of EECS, QMUL. \\
\years{07/2015--09/2017}\textsc{Postdoctoral Fellow}. Center for Music Technology (GTCMT) | Digital Media Program, Georgia Tech. \\
\years{08/2013--09/2014}\textsc{Research Fellow}. London Knowledge Lab, UCL Institute of Education. London. \\
%{\small Qualitative data collection and analysis (6 sites), development of research tools and processes, dissemination activities and paper writing of results.}\\
\years{02/2004--06/2010}\textsc{Co-Founder, Project Manager, Web Designer \& Web Developer}. Nodular Soft. Barcelona. \\
%{\small Freelance studio focused on user-centric software and AV communication, development of community websites using several CMS, development of AV programs under specific needs, and usability consultancy.} \\
\years{01/2008--07/2009}\textsc{Web Designer \&  Web Developer Project Officer}. Music Technology Group, UPF. Barcelona. \\
%{\small Web design and web development of the web 2.0 Sons de Barcelona (barcelona.freesound.org). Web design of the corporate portal of the research group MTG (mtg.upf.edu). Graphic design of the corporate brochure of the MTG (courses 2008--2009 and 2009--2010).}\\
\years{11/2007--06/2009}\textsc{Web Designer \&  Web Developer Project Officer}. Uaalah!!. Barcelona. \\
%{\small Design and programming of a self-manageable interactive catalogue for CD-ROM about Bürkert products. Flash programming of the online tea shop of Sans \& Sans (sansisans-finetea.com).}\\
\years{08/2005--09/2006}\textsc{Web Designer \&  Motion Graphic Designer}. CCRTVi | TV3 Interactiva. Sant Just Desvern, Barcelona. \\
%{\small Web interface design of different portals of the Catalan TV corporation (tv3.cat, catradio.cat, ritmes.net, among others). Web interface design of the sitcom Lo Cartanyà (locartanya.com). Mobile design of the prototype 3alacarta.}\\
\years{05/2001--08/2002}\textsc{Web Designer \&  Motion Graphic Designer}. TerraNetworks | UranoFilms. Barcelona. \\
%{\small Web interface design and flash design of AV internet content about electronic music and digital culture.}\\
\years{04/2000--05/2001}\textsc{Web Designer \&  Motion Graphic Designer}. MediaPark | ParkNet, Barcelona. %[1.3cm]%hack
%{\small Flash design of animations games for the Internet soccer portal futvol.com.} \\[1.1cm]

%\hrule
\section*{Honors \& Awards}
%\noindent

\subsection*{Academic Grants, Honors, Awards \& Accolades}
\years{10/2021--06/2022} \textsc{Future Research Leaders (FRL) 8 Programme} accolade. DMU, Leicester, UK.\\
\years{2019c} \textsc{Web Audio Conference 2019 Best Paper Award} to Anna Xambó, Robin Støckert, Alexander Refsum Jensenius and Sigurd Saue. NTNU, Trondheim, Norway.\\
\years{2019b} \textsc{The EUNIS DØRUP E-Learning Award} to Robin Støckert, Alexander Refsum Jensenius, Anna Xambó and Øyvind Brandtsegg. NTNU, Trondheim, Norway.\\
%\years{2019a} \textsc{Travel grant for European research and promotion} (30,000 kr, approx. 3,000 \euro). NTNU, Trondheim, Norway.\\
%\years{2018--2019} \textsc{3-year Startup Grant (Startpakke)} (250,000 kr, approx. 25,000 \euro), awarded to women in permanent academic positions in a scientific field with underrepresentation of women. NTNU, Trondheim, Norway.\\
\years{2017} NCWIT Engagement Excellence Award (\$5,000 cash award) to Greg Hendler, Léa Ikkache, Brandon Westergaard, Anna Xambó, Doug Edwards, Brian Magerko, and Jason Freeman (Earsketch), Georgia Tech.\\
\years{2016}\textsc{Women in Music Information Retrieval (WiMIR) grant}, awarded conference fee waiver to attend the ISMIR 2016 conference. New York University (NYU), New York.\\
\years{10/2010--07/2013}\textsc{3-year Fully-Funded Full-Time OU PhD scholarship}. The Open University, Milton Keynes, UK.\\
\years{03/2010--06/2010}\textsc{4-month Fully-Funded OU Visiting Research Studentship}. The Open University, Milton Keynes, UK.

\subsection*{Artistic Honors \& Awards}
\years{05/2004}\textsc{First prize award Minima Festival}. Gandía, Spain. \\
Category: Experimental Video. \\
Project: ``Cosmogonias". \\
Role: Creator \& Director.

%\hrule
\section*{Grants \& Funding}
%\noindent

\subsection*{Principal Investigator}

\years{11/2021--06/2022}\textsc{Future Research Leaders 8 -- Development fund} \\
Funding body: DMU. \\
Project: Development fund that is part of the FRL Programme.\\
Role: Lead Researcher/Coordinator. \\
Total Pound Amount: \pounds1,000 \\

\years{04/2020-10/2021}\textsc{Human Data Interaction Network Plus/EPSRC Grant} \\
Funding body: EPSRC. \\
Project: ``MIRLCAuto: A Virtual Agent for Music Information Retrieval in Live Coding''.\\
Role: PI. \\
Partners: Music, Technology and Innovation - Institute for Sonic Creativity (MTI$^2$), De Montfort University; Leicester Hackspace; IKLECTIK; Phonos; l’ull cec. \\
Collaborators: TOPLAP Barcelona, FluCoMa, Freesound.\\
Total Pound Amount: \pounds10,000 \\

\years{08/2018-12/2019}\textsc{Start-up Package (Startpakke)} \\
Funding body: NTNU. \\
The start-up package is awarded to women in permanent academic positions to help them start their academic careers within a scientific field that has an underrepresentation of women. With part of this funding, I co-founded and was chair of the organisation Women Nordic in Music Technology (\emph{WoNoMute}) (see Entrepreneurship). \\
Role: PI. \\
Total Norwegian Krone Amount: kr250,000, approx. \euro25,000 \\

\years{11/2003-10/2004}\textsc{Teaching innovation project grant} \\
Funding body: Fundació Caixa de Sabadell. \\
Project: ``Crossmedia infantil: Estudio sobre las nuevas tecnologías y la comunicación audiovisual en la escuela infantil y primaria (Crossmedia for Children: New Technologies and Audiovisual Communication in Primary Education).''\\
Role: PI. \\
Collaborators: Eladi Martos (Co-PI), UB. \\
Total Euro Amount: \euro3,000
%Candidate’s Share: 50\% (1,500 \euro)

\subsection*{Collaborator}

\years{09/2016-08/2020}\textsc{Advancing Informal STEM Learning Grant} \\
Funding body: National Science Foundation (NSF). \\
Project: ``Collaborative Research: Mixing Learning Experiences for Computer Programming Across Museums, Classrooms, and the Home Using Computational Music''. Award Number: 1612644. \\
Organization: Georgia Tech Research Corporation. \\
Role: Postdoctoral Fellow and Co-Writer of the grant proposal. \\
Collaborators: Brian Magerko (PI), Jason Freeman (Co-PI), Mike Horn (Co-PI).\\
Total Dollar Amount: \$2,517,690.00 

\subsection*{Fundraiser}

\years{2019b}\textsc{Web Audio Conference 2019} \\
Role: Conference chair. \\
Total Fundraised Norwegian Krone Amount: kr120,000 (approx. \euro12,000)\\[.1cm]
\years{2019a}\textsc{Internal funding for a COST application} \\
Role: Project leader. \\
Travel grant for European research and promotion.\\
Total Fundraised Norwegian Krone Amount: kr30,000 (approx. \euro3,000)\\[.1cm]
\years{2017--2018}\textsc{Female Laptop Orchestra + WoNoMute +YSWN + Sonora: seminar and concert at SARC} \\
Role: Co-organizer and participant of the event. \\
Total Fundraised British Pound Amount: \pounds2,000 \\[.1cm]
\years{05/2016--05/2017}\textsc{Women in Music Tech} \\
Role: Co-Founder \& Co-Chair of the organization. \\
Total Fundraised Dollar Amount: \$11,450 
%\years{08/2016--05/2017}\emph{2016-2017 Academic Year}\\
%Funding body: School of Music, Georgia Tech. \\
%Total Dollar Amount: \$2,500\\[.1cm]
%\years{08/2016--11/2016}\emph{Fall 2016 Concert Event}\\
%Funding body: College of Design, Georgia Tech. \\
%Total Dollar Amount: \$2,000\\[.1cm]
%Funding body: ADVANCE program, Georgia Tech. \\
%Total Dollar Amount: \$1,000\\[.1cm]
%Funding body: Women's Resource Center, Georgia Tech. \\
%Total Dollar Amount: \$250\\[.1cm]
%\years{01/2017--05/2017}\emph{Spring 2017 Actions}\\
%Funding body: School of Music, Georgia Tech. \\
%Total Dollar Amount: \$2,400\\[.1cm]
%Funding body: College of Design Council Diversity, Georgia Tech. \\
%Total Dollar Amount: \$1,500\\[.1cm]
%Funding body: ADVANCE Program, Georgia Tech. \\
%Total Dollar Amount: \$1,000\\[.1cm]
%Funding body: Women's Resource Center, Georgia Tech. \\
%Total Dollar Amount: \$500\\[.1cm]
%Funding body: Digital Media Program, School of Literature, Media, and Communication, Georgia Tech. \\
%Total Dollar Amount: \$300

\subsection*{Creator | Director}

\years{09/2001--08/2002}\textsc{Audiovisual production grant} \\
Funding body: Departament de Cultura de la Generalitat de Catalunya (Department of Culture of Catalan Government).\\
Project: ``Transdata Pr.''.  \\
Role: Creator, Director \& Video Editor.\\
Collaborators: Gerard Roma (music), Oscar Abril Ascaso (essay). \\
Total Euro Amount: \euro3,000\\ 
%Candidate’s Share: 50\% (1,500 \euro)\\[0.5cm]%hack

\years{09/1998--08/1999}\textsc{Audiovisual production grant} \\
Funding body: Departament de Cultura de la Generalitat de Catalunya (Department of Culture of Catalan Government).\\
Project: ``Mitösömä''.  \\
Role: Creator, Director \& Animation Editor.\\
Collaborators: Gerard Roma (music). \\
Total Euro Amount: \euro3,000 
%Candidate’s Share: 50\% (1,500 \euro)



\section*{Research Profiles}
\noindent

\textbullet \- \href{http://open.academia.edu/AnnaXambo}{Academia.edu}\\
%\textbullet \- \href{https://app.cristin.no/persons/show.jsf?id=997893}{CRISTIN (Current Research Information System in Norway)}\\
\textbullet \- \href{https://dora.dmu.ac.uk/browse?type=author&value=Xambo,\%20Anna}{DORA (De Montfort Open Research Archive)}\\
\textbullet \- \href{https://scholar.google.com/citations?user=yi3WXM8AAAAJ}{Google Scholar}\\
\textbullet \- \href{http://oro.open.ac.uk/view/person/ax22.html}{Open Research Online}\\
\textbullet \- \href{https://orcid.org/0000-0003-2333-6941}{ORCID}\\
\textbullet \- \href{http://www.researchgate.net/profile/Anna_Xambo}{ResearchGate}\\
\textbullet \- \href{https://www.scopus.com/authid/detail.uri?authorId=36642886000}{SCOPUS}\\
\textbullet \- \href{https://www.semanticscholar.org/author/Anna-Xamb\%C3\%B3/1915531}{Semantic Scholar}\\
\textbullet \- \href{https://publons.com/researcher/3980818/anna-xambo}{Web of Science}
%[1.2cm]%hack

\section*{Publications}
%\noindent

\subsection*{Books / Edited Books}
\noindent

\years{2020}\textbf{Xambó, A.},  Martín, S. R., Roma, G. (eds.) (2019). \emph{Proceedings of the International Web Audio Conference}. NTNU. ISSN: 2663-5844.\\
\years{2019}Queiroz, M. \textbf{Xambó, A.} (eds.) (2019). \emph{Proceedings of the International Conference on New Interfaces for Musical Expression}. UFRGS. ISSN 2220-4806.\\
\years{2004}\textbf{Xambó, A.} (2004). \emph{Herramientas De Diseño Digital / Digital Design Tools}. Madrid: Anaya-Multimedia. Editor: Eugenio Tuya Feijoo. ISBN 8441516979.

\subsection*{Peer-Reviewed Book Chapters}
\noindent

\years{2020}Støckert, R., Bergsland, A., \textbf{Xambó, A.} (2020). “The Notion of Presence in a Telematic Cross-Disciplinary Program for Music, Communication and Technology". In Eiksund, Ø. J., Angelo, E., Knigge, J. eds. Music Technology in Education -- Channeling and Challenging Perspectives. Cappelen Damm Akademisk, Oslo. pp. 77--101.\\
\years{2019}\textbf{Xambó, A.}, Font, F., Fazekas, G., Barthet, M. (2019). “Leveraging Online Audio Commons Content For Media Production". In Michael Filimowicz ed. Foundations in Sound Design for Linear Media: An Interdisciplinary Approach. Routledge. pp. 248-282. ISBN 9781138093966.\\
\years{2016}\textbf{Xambó, A.} (2017), “Embodied Music Interaction: Creative Design Synergies Between Music Performance and HCI". In Price, S. and Broadhurst, S. eds. Digital Bodies: Creativity and Technology in the Arts and Humanities. Palgrave Macmillan, London. pp. 207--220. ISBN 9781349952410.\\
\years{2013}\textbf{Xambó, A.}, Laney, R., Dobbyn, C., Jordà, S. (2013). “Video Analysis for Evaluating Music Interaction: Musical Tabletops". In Holland, S., Wilkie, K., Mulholland, P. and Seago, A. eds. Music and Human-Computer Interaction. Springer, London. pp. 241--258. ISBN 9781447129905.


\subsection*{Peer-Reviewed Journal Special Issues}
\noindent
\years{2023}\textbf{Xambó, A.}, Roma, G., Magnusson, T. (eds.) (Forthcoming). Live Coding Sonic Creativities. \emph{Organised Sound}, 28(2).\\
\years{2020}\textbf{Xambó, A.},  Martín, S. R., Roma, G. (eds.) (2020). JAES Special Issue on Web Audio. \emph{Journal of Audio Engineering Society}, 68(10).


\subsection*{Journal Articles}
\noindent
{\years{2021b}\textbf{Xambó, A.} (Submitted) ``Virtual Agents in Live Coding: A Short Review''. \emph{e-Contact!}, (online journal).\\
{\years{2021a}Roma, G., \textbf{Xambó, A.}, Green, O., Tremblay, P.A. (2021) ``A General Framework for Visualization of Sound Collections in Musical Interfaces''. \emph{Applied Sciences}. 11(24):11926.\\
{\years{2020a}\textbf{Xambó, A.}, Støckert, R., Jensenius, A.R. and Saue, S. ``Learning to Code Through Web Audio: A Team-Based Learning Approach''. \emph{Journal of Audio Engineering Society}, 68(10), pp. 727--737. \emph{Special Issue on Web Audio}.}\\
\years{2019}\textbf{Xambó, A.}, Lerch, A., Freeman, J. “Music Information Retrieval in Live Coding: A Theoretical Framework". \emph{Computer Music Journal}, 42(4), Winter 2018, pp. 9--25.\\
\years{2018b}Roma, G. and \textbf{Xambó, A.}, Freeman, J. (2018). “User-independent Accelerometer Gesture Recognition for Participatory Mobile Music". \emph{Journal of Audio Engineering Society}, 66(6), pp. 430--438.\\
\years{2018a}\textbf{Xambó, A.}, Roma, G., Shah, P., Tsuchiya, T., Freeman, J., Magerko, B. (2018). “Turn-taking and Online Chatting in Co-located and Remote Collaborative Music Live Coding". \emph{Journal of Audio Engineering Society}, 66(4), pp. 253--256.\\
\years{2017c}\textbf{Xambó, A.}, Hornecker, E., Marshall, P., Jordà, S., Dobbyn, C., Laney, R. (2017). “Exploring Social Interaction with a Tangible Music Interface". \emph{Interacting with Computers}, 29(2), pp. 248--270.\\
\years{2017b}Jewitt, C., Price, S., \textbf{Xambó, A.} (2017). “Conceptualising and Researching the Body in Digital Contexts: Towards New Methodological Conversations Across the Arts and Social Sciences". \emph{Qualitative Research}, 17(1), pp. 37--53.\\
\years{2017a}Jewitt, C., \textbf{Xambó, A.}, Price, S. (2017). “Exploring Methodological Innovation in the Social Sciences: The Body in Digital Environments and the Arts". \emph{International Journal of Social Research Methodology},  20(1), pp. 105--120.\\
\years{2013b}\textbf{Xambó, A.}, Hornecker, E., Marshall, P., Jordà, S., Dobbyn, C., Laney, R. (2013). “Let's Jam the Reactable: Peer Learning during Musical Improvisation with a Tabletop Tangible Interface". \emph{ACM Transactions on Computer-Human Interaction}, 20(6), pp. 36:1--36:34.\\
\years{2013a}Bogdanov, D., Haro, M., Fuhrmann, F., \textbf{Xambó, A.}, Gómez, E., Herrera, P. (2013). “Semantic Audio Content-based Music Recommendation and Visualization based on User Preference Examples". \emph{Information Processing \& Management}, 49(1), pp. 13--33.%\\[1.2cm]%hack

\subsection*{Peer-Reviewed Conference Papers}
\noindent

{\years{2021}\textbf{Xambó, A.}, Roma, G., Roig, S., Solaz, E. (2021) ``Live Coding with the Cloud and a Virtual Agent''. \emph{Proceedings of the New Interfaces for Musical Expression (NIME '21)}. Shanghai, China.}\\
{\years{2020b}\textbf{Xambó, A.}, Roma, G. (2020) ``Performing Audiences: Composition Strategies for Network Music using Mobile Phones''. \emph{Proceedings of the New Interfaces for Musical Expression (NIME '20)}. Birmingham, UK. pp. 55--60.}\\
\years{2020a}Jawad, K., \textbf{Xambó, A.} (2020) ``How to Talk of Music Technology: An Interview Analysis Study of Live Interfaces for Music Performance among Expert Women''. In \emph{Proceedings of the International Conference on Live Interfaces (ICLI 2020)}. Trondheim, Norway. pp. 41--47.\\
\years{2019b}\textbf{Xambó, A.}, Støckert, R., Jensenius, A.R. and Saue, S. (2019) ``Facilitating Team-Based Programming Learning with Web Audio''. In \emph{Proceedings of the Web Audio Conference 2019 (WAC '19)}. Trondheim, Norway. pp. 2--7. \emph{Best Paper Award}.\\
\years{2019a}\textbf{Xambó, A.}, Saue, S., Jensenius, A.R., Støckert, R., Brandtsegg, Ø. (2019) "NIME Prototyping in Teams: A Participatory Approach to Teaching Physical Computing". In \emph{Proceedings of the New Interfaces for Musical Expression (NIME ’19)}. Porto Alegre, Brazil. pp. 216–221.\\
\years{2018f}Roma, G., \textbf{Xambó, A.}, Green, O., Tremblay, P.A. (2018) "A Javascript Library for Flexible Visualization of Audio Descriptors". In \emph{Proceedings of the Web Audio Conference (WAC '18)}. Berlin, Germany.\\
\years{2018e}Pauwels, J., \textbf{Xambó, A.}, Roma, G., Barthet, M. Fazekas, G. (2018) "Exploring Real-time Visualisations to Support Chord Learning with a Large Music Collection". In \emph{Proceedings of the Web Audio Conference (WAC '18)}. Berlin, Germany.\\
\years{2018d}\textbf{Xambó, A.}, Pauwels, J., Roma, G., Barthet, M. Fazekas, G. (2018) "Jam with Jamendo: Querying a Large Music Collection by Chords from a Learner’s Perspective". In \emph{Proceedings of Audio Mostly 2018: Sound in Immersion and Emotion (AM '18)}. Wrexham, United Kingdom. \\
\years{2018c}\textbf{Xambó, A.} (2018) “Who Are the Women Authors in NIME?---Improving Gender Balance in NIME Research". In \emph{Proceedings of the New Interfaces for Musical Expression (NIME '18)}. Blacksburg, Virginia, USA. pp. 174--177.\\ 
\years{2018b}\textbf{Xambó, A.}, Roma, G., Lerch, A., Barthet, M., Fakekas, G. (2018) “Live Repurposing of Sounds: MIR Explorations with Personal and Crowdsourced Databases". In \emph{Proceedings of the New Interfaces for Musical Expression (NIME '18)}. Blacksburg, Virginia, USA. pp. 364--369.\\ 
\years{2018a}Weisling, A., \textbf{Xambó, A.}, Olowe, I., Barthet, M. (2018) “Surveying the Compositional and Performance Practices of Audiovisual Practitioners". In \emph{Proceedings of the New Interfaces for Musical Expression (NIME '18)}. Blacksburg, Virginia, USA. pp. 344--345.\\ 
\years{2017d}\textbf{Xambó, A.}, Shah, P., Roma, G., Freeman, J., Magerko, B. (2017) “Turn-taking and Chatting in Collaborative Music Live Coding". \emph{Nominee for Best Paper Award}. In \emph{Proceedings of the Audio Mostly Conference (AM '17)}. London.\\ 
\years{2017c}Roma, G., \textbf{Xambó, A.}, Freeman, J. (2017) “Handwaving: Gesture Recognition for Participatory Mobile Music". In \emph{Proceedings of the Audio Mostly Conference (AM '17)}. London.\\
\years{2017b}Roma, G., \textbf{Xambó, A.}, Freeman, J. (2017) “Loop-aware Audio Recording for the Web". In \emph{Proceedings of the Web Audio Conference 2017 (WAC '17)}. London.\\ 
\years{2017a}\textbf{Xambó, A.}, Drozda, B., Weisling, A., Magerko, B., Huet, M., Gasque, T., Freeman, J. (2017) Experience and Ownership with a Tangible Computational Music Installation for Informal Learning. In \emph{Proceedings of the Tangible, Embedded, and Embodied Interaction Conference (TEI '17)}. Yokohama, Japan. pp. 351--360.\\ 
\years{2016c} Moore, R., Edwards, D., Freeman, J., Magerko, B., McKlin, T., \textbf{Xambó, A.} (2016). “EarSketch: An Authentic, STEAM-Based Approach to Computing Education". In \emph{Proceedings of the 2016 American Society for Engineering Education Annual Conference \& Expo}. New Orleans, Louisiana.\\
\years{2016b}Freeman, J., Magerko, B., Edwards, D., Miller, M., Moore, R., \textbf{Xambó, A.} (2016). “Using EarSketch to Broaden Participation in Computing and Music". In \emph{Proceedings of the 13th Sound and Music Computing Conference (SMC 2016)}. Hamburg, Germany. pp. 156--163.\\
\years{2016a}\textbf{Xambó, A.}, Freeman, J., Magerko, B., Shah, P. (2016). “Challenges and New Directions for Collaborative Live Coding in the Classroom". In \emph{Proceedings of the International Conference on Live Interfaces (ICLI 2016)}. Brighton, UK. pp. 65--73.\\
\years{2014}\textbf{Xambó, A.}, Roma, G., Laney, R., Dobbyn, C. and Jordà, S. (2014). “SoundXY4: Supporting Tabletop Collaboration and Awareness with Ambisonics Spatialisation". In \emph{Proceedings of the International Conference on New Interfaces for Musical Expression 2014 (NIME '14)}. London. pp. 249--252.\\
\years{2013}Bogdanov, D., Haro, M., Fuhrmann, F., \textbf{Xambó, A.}, Gómez, E. and Herrera, P. (2013). “A Content-based System for Music Recommendation and Visualization of User Preferences Working on Semantic Notions". In \emph{IEEE 9th International Workshop on Content-Based Multimedia Indexing (CBMI '13)}. Madrid. pp. 249--252.\\
\years{2012}Roma, G., \textbf{Xambó, A.}, Herrera, P. and Laney, R. (2012). “Factors in human recognition of timbre lexicons generated by data clustering". In \emph{Proceedings of the 9th Sound and Music Computing Conference (SMC 2012)}. Copenhagen, Denmark. pp. 23--30.\\
\years{2011c}\textbf{Xambó, A.}, Laney, R., Dobbyn, C. and Jordà, S. (2011). “Multi-touch Interaction Principles for Collaborative Real-time Music Activities: Towards a Pattern Language". In \emph{Proceedings of the International Computer Music Conference (ICMC '11)}. Huddersfield, UK. pp. 403--406.\\
\years{2011b}\textbf{Xambó, A.}, Laney, R. and Dobbyn, C. (2011). “TOUCHtr4ck: Democratic Collaborative Music". In \emph{Proceedings of the Tangible, Embedded, and Embodied Interaction Conference (TEI '11)}. Funchal, Madeira. pp. 309--312.\\
\years{2011a}Milne, A. J., \textbf{Xambó, A.}, Laney, R., Sharp, D. B., Prechtl, A. and Holland, S. (2011). “Hex Player — A Virtual Musical Controller". In \emph{Proceedings of the International Conference on New Interfaces for Musical Expression (NIME '11)}. Oslo, Norway. pp. 244--247.\\
\years{2010b}Laney, R., Dobbyn, C., \textbf{Xambó, A.}, Schirosa, M., Miell, D., Littleton, K. and Dalton, N. (2010). “Issues and Techniques for Collaborative Music Making on Multi-touch Surfaces". In \emph{Proceedings of the 7th Sound and Music Computing Conference (SMC 2010)}. Barcelona. pp. 146–153.\\
\years{2010a}Haro, M., \textbf{Xambó, A.}, Fuhrmann, F., Bogdanov, D., Gómez, E. and Herrera, P. (2010). “The Musical Avatar: A Visualization of Musical Preferences by means of Audio Content Description". In \emph{Proceedings of the 5th Audio Mostly Conference (AM '10)}. Piteå, Sweden.\\
\years{2008}Roma, G. and \textbf{Xambó, A.} (2008). “A Tabletop Waveform Editor for Live Performance". In \emph{Proceedings of the International Conference on New Interfaces for Musical Expression (NIME '08)}. Genoa, Italy.

\subsection*{Peer-Reviewed Abstracts with Proceedings}
\noindent
\years{2021b}\textbf{Xambó, A.} (2021). ``Live Coding with Crowdsourced Sounds and A Virtual Agent Companion''. In \emph{Proceedings of the Web Audio Conference 2021 (WAC '21)}. Barcelona, Spain.\\
\years{2021a}\textbf{Xambó, A.} (2021). ``A Live Coding Session With the Cloud and a Virtual Agent''. In \emph{Proceedings of the New Interfaces for Musical Expression (NIME '21)}. Shanghai, China.}\\
\years{2019b}Brandtsegg, Ø., \textbf{Xambó, A.}, Engum, T., Bergsland, A., Waadeland, C.H. (2019). ``Trondheim EMP Repository processing''. In \emph{Proceedings of the Web Audio Conference 2019 (WAC '19)}. Trondheim, Norway. pp. 161--162.\\
\years{2019a}Støckert, R., Jensenius, A.R., \textbf{Xambó, A.}, Brandtsegg, Ø. (2019). ``A Case Study in Learning Spaces for Physical-Virtual Two-Campus Interaction''. In \emph{Proceedings of the European University Information Systems (EUNIS)}. \emph{Dørup Award winning paper}.\\ 
Brandtsegg, Ø., \textbf{Xambó, A.}, Engum, T., Bergsland, A., Waadeland, C.H. (2019). ``Trondheim EMP Repository processing''. In \emph{Proceedings of the Web Audio Conference 2019 (WAC '19)}. Trondheim, Norway. pp. 161--162.\\
\years{2018b}Skach, S., \textbf{Xambó, A.}, Turchet, L., Stolfi, A., Stewart, B., Barthet, M. (2018). “Embodied Interactions with E-Textiles and the Internet of Sounds for Performing Arts". In \emph{Proceedings of the Twelfth International Conference on Tangible, Embedded, and Embodied Interaction (TEI '18)}. Stockholm, Sweden. pp. 80--87.\\
\years{2018a}Weisling, A., \textbf{Xambó, A.} (2018). “Beacon: Exploring Physicality in Digital Performance". In \emph{Proceedings of the Twelfth International Conference on Tangible, Embedded, and Embodied Interaction (TEI '18)}. Stockholm, Sweden. pp. 586--591.\\
\years{2017a}\textbf{Xambó, A.}, Roma, G. (2017). “Hyperconnected Action Painting". In \emph{Proceedings of the Web Audio Conference 2017 (WAC '17)}. London.\\
\years{2016c}Tsuchiya, T., \textbf{Xambó, A.}, Freeman, J. (2016). “Adapting DAW-driven Musical Language to Live Coding: A Case Study in EarSketch". In \emph{Late-Breaking Demo of the Second International Conference on Live Coding (ICLC '16)}. Hamilton, Canada.\\ 
\years{2016b}\textbf{Xambó, A.}, Lerch, A., Freeman, J. (2016). “Learning to Code Through MIR". In \emph{Extended Abstracts for the Late-Breaking Demo Session of the 17th International Society for Music Information Retrieval Conference (ISMIR 2016)}. New York.\\
\years{2016a}Roma, G., \textbf{Xambó, A.}, Freeman, J. (2016). “Do the Buzzer Shake". In \emph{Proceedings of the International Conference on Live Interfaces (ICLI 2016)}. Brighton, UK. pp. 315--316.\\
\years{2015}Freeman, J., Magerko, B., Edwards, D., Moore, R., McKlin, T., \textbf{Xambó, A.} (2015). “EarSketch: A STEAM Approach to Broadening Participation in Computer Science Principles". In \emph{Proceedings of the IEEE Research in Equity and Sustained Participation in Engineering, Computing, and Technology (RESPECT '15)}. Charlotte, NC. pp. 109--110.\\
\years{2014}\textbf{Xambó, A.}, Jewitt, C., and Price, S. (2014). “Towards an Integrated Methodological Framework for Understanding Embodiment in HCI". In \emph{Proceedings of the Extended Abstracts on Human Factors in Computing Systems (CHI '14)}. Toronto. pp. 1411--1416.

\subsection*{Position \& Workshop Papers}
\noindent

\years{2020}Fasciani, S., Jensenius, A.R., Støckert, R., \textbf{Xambó, A.} (2020) ``The MCT Portal: An Infrastructure, a Laboratory and a Pedagogical Tool''. Researchers’ Colloquium on Low Latency Streaming in Music Learning and Teaching. Royal Conservatoire of Scotland, Glasgow, UK.\\
\years{2017}\textbf{Xambó, A.}, Roma, G., Shah, P., Freeman, J., Magerko, B. (2017) “Computational Challenges of Co-creation in Collaborative Music Live Coding: An Outline". 2017 Co-Creation Workshop at the International Conference on Computational Creativity. Atlanta, GA, USA.\\ 
\years{2012}\textbf{Xambó, A.}; Laney, R.; Dobbyn, C. and Jordà, S. (September 11, 2012). “Towards a Taxonomy for Video Analysis on Collaborative Musical Tabletops". In \emph{BCS HCI 2012 Workshop on Video Analysis Techniques for HCI}. Birmingham, UK.\\
\years{2011}\textbf{Xambó, A.}; Laney, R.; Dobbyn, C. and Jordà, S. (July 4, 2011). ``Collaborative Music Interaction on Tabletops: An HCI Approach'''. In \emph{BCS HCI 2011 Workshop on When Words Fail: What can Music Interaction tell us about HCI?}. Newcastle Upon Tyne.

\subsection*{Dissertation}
\noindent

\years{2015}\textbf{Xambó, A.} (2015). ``Tabletop Tangible Interfaces for Music Performance: Design and Evaluation''. PhD thesis. The Open University. Milton Keynes, UK.

\subsection*{Reports \& Working Papers}
\noindent

\years{2008}\textbf{Xambó, A.} (2008). ``Interfaces for Sketching Musical Compositions''. Master thesis. Unpublished. UPF. Barcelona, Spain\\ 
\years{2004}\textbf{Xambó, A.}, Martos, E. (2004). ``Crossmedia Infantil: Estudi sobre les noves tecnologies i la comunicació audiovisual a l'escola infantil i primària'' (Report of New Technologies and Audiovisual Communication in the Primary Education). Report. Unpublished. Supported by Fundació Caixa de Sabadell. In collaboration with UB. Barcelona, Spain.

\subsection*{Book Reviews}
\noindent

\years{2021}\textbf{Xambó, A.} (2021). ``Book Review: \emph{Thor Magnusson, Sonic Writing: Technologies of Material, Symbolic, and Signal Inscriptions. New York: Bloomsbury Academic, 2019. ISBN: 9781501313851. DOI: https://doi.org/10.5040/9781501313899}''. \emph{Organised Sound}, 26:3, pp. 430–433.\\ 


\section*{Talks, Panels \& Oral Presentations}
%\noindent

\subsection*{External}
\noindent

\years{2021h} \textsc{Keynote Speaker (online)} (July 5, 2021). \emph{Web Audio Conference 2021 (WAC 2021)}. Barcelona, Spain. \href{https://youtu.be/7SWu3txbg-w}{[video]}\\ 
\years{2021g} \textsc{Oral Presenter (online)} together with Jawad, K.  (June 19, 2021). "Challenging the Status Quo - WoNoMute". \emph{Féminisme - Musique - Technologie / Rencontre ManiFeste-2021}. Centre Wallonie-Bruxelles, Paris, France. \href{https://youtu.be/jXmNvd9ty_o}{[video]}\\
\years{2021f} \textsc{Oral Presenter (online)} (June 15, 2021). “Live Coding with the Cloud and a Virtual Agent". \emph{NIME '21}. 
\emph{NYU Shanghai}, Shanghai, China. \href{https://youtu.be/F4UoH1hRMoU}{[video]}\\
\years{2021e} \textsc{Oral Presenter (online)} (May 21, 2021). “Audience engagement in musical performances through on-site and online networks". \emph{Computer Networks meet Music Instruments}. COSY Colloquium, Faculty of Computer Science, University of Vienna. Vienna, Austria.\\ 
\years{2021d} \textsc{Panel Member (online)} (May 6, 2021). “Session 3: Musical performance \& education with AI" with Rafael Ramírez (Universitat Pompeu Fabra), George Waddell (Royal College of Music), Anna Xambó (De Montfort University), Luisa Pereira (New York University), Enric Guaus (Escola Superior de Música de Catalunya), Alia Ahmed Morsi (Universitat Pompeu Fabra) and moderated by Sergio Giraldo (Universitat Pompeu Fabra). \emph{ARTIFICIA Festival}. Barcelona, Spain. \href{https://youtu.be/o0arHV4s6Mo}{[video]}\\ 
\years{2021c} \textsc{Oral Presenter (online)} (April 29, 2021). “Insights Into MIRLCAuto: A Virtual Agent for Music Information Retrieval in Live Coding". \emph{Coding Literacy, Practices and Cultures / Colloquium}. University of Magdeburg, Germany, and Film University KONRAD WOLF in Potsdam Babelsberg, Germany.\\ 
\years{2021b} \textsc{Panel Member (online)} (April 16, 2021). “MusicLab 6: Human-Machine Improvisation" with Stefania Serafin (Aalborg University), Anna Xambó Sedó (De Montfort University), Nanette Nielsen (RITMO) and the performers, Christian Winther and Dag Erik Knedal Andersen, moderated by Alexander Refsum Jensenius (RITMO). \emph{RITMO Centre for Interdisciplinary Studies in Rhythm, Time and Motion}. Oslo, Norway. \href{https://youtu.be/yeIRxkm-kSc}{[video]}\\ 
\years{2021a} \textsc{Oral Presenter (online)} (February 23, 2021). “Live Coding Using Crowdsourced Sounds and a Virtual Agent". \emph{ATLAS Seminar / Colloquium}. University of Colorado Boulder, Boulder, CO, USA.\\ 
\years{2020c} \textsc{Oral Presenter (online)} (October 6, 2020). “From Tabletop Tangible Interfaces to Virtual Agents in Live Coding: Five Years of Research in Progress". \emph{IUPUI MAT PhD seminar}. Department of Music and Arts Technology, Purdue School of Engineering and Technology, Indiana University--Purdue University Indianapolis, Indianapolis, IN, USA.\\ 
\years{2020b} \textsc{Oral Presenter (online)} (September 21, 2020). “Virtual Agents in Live Coding: Preliminary Investigations". \emph{Meetup for Women in Arts and Technology}. NOTAM, Oslo, Norway.\\ 
\years{2020a} \textsc{Keynote Speaker (online)} (June 24, 2020). “Collaborative/Participatory Music Experiences: A Dialogue Between SMC and HCI". \emph{Sound and Music Computing Conference 2020 (SMC 2020)}. Torino, Italy. \href{https://youtu.be/fWqhhdkVO6o}{[video]}\\ 
\years{2019d} \textsc{Oral Presenter} (December 5, 2019). “Facilitating Team-Based Programming Learning with Web Audio". \emph{WAC '19}. 
\emph{NTNU}, Trondheim, Norway.\\
\years{2019c} \textsc{Oral Presenter} (November 22, 2019). “Research as Practice / Practice as Research: Real-time Algorithmic Music Experiences and Women in Music Technology". \emph{NOTAM}, Oslo, Norway.\\
\years{2019b} \textsc{Oral Presenter} together with Jawad, K. and Shrestha, S. (June 20, 2019). “International joint Master's programme in Music, Communication \& Technology: A master for technological humanists". \emph{MA Creative Technologies}. \emph{Filmuniversität  Babelsberg Konrad Wolf}, Potsdam, Germany.\\
\years{2019a} \textsc{Oral Presenter} (July 5, 2019). "NIME Prototyping in Teams: A Participatory Approach to Teaching Physical Computing". \emph{NIME '19}. Centro cultural, Universidade Federal do Rio Grande do Sol (UFRGS), Porto Alegre, Brazil.\\
\years{2018j}\textsc{Panel Member}. (November 21, 2018). \emph{Panel: Women in Music Technology around the World} with Nela Brown (FLO), Magdalena Chudy (FLO), Liz Dobson (YSWN), Ada Mathea Hoel (WoNoMute), Léa Ikkache (WiMT), Tuna Pase (FLO), Franziska Schroeder (FLO), Ariane Stolfi, (Sonora) Sonia Wilkie (FLO) and Anna Xambó (WoNoMute, WiMT). Sonic Arts Research Center, Queen's University Belfast. Belfast, Northern Ireland.\\
\years{2018i}\textsc{Panel Member}. (November 15, 2018). \emph{Panel: Future of the Music Industries} with Joe Lyske (chair, MXX), Jesper Skibsby (panelist, WARM), Nick Breen (panelist, Reed Smith) and Anna Xambó (panelist, NTNU). Resonate Music Conference 2018. Barras Art and Design (BAAD), Glasgow, Scotland, UK.\\
\years{2018h}\textsc{Panel Member}. (October 26, 2018). \emph{Panel Session 3: Equality, Diversity, Gender} with Thomas Hilder (chair), Jill Diana Halstead Hjørnevik (panelist), Sunniva Skjøstad Hovde (panelist), Vivian Anette Lagesen (panelist), and Anna Xambó (panelist). Knowing Music -- Musical Knowing: Cross disciplinary dialogue on epistemologies. International Music Research School 2018, NTNU. Dokkhuset, Trondheim, Norway.\\
\years{2018g} \textsc{Oral Presenter} together with Pauwels, J.  (September 19, 2018). "Exploring Real-time Visualisations to Support Chord Learning with a Large Music Collection". \emph{WAC '18}. \emph{Technische Universität Berlin}, Berlin, Germany.\\
\years{2018f} \textsc{Oral Presenter} together with Pauwels, J. (September 14, 2018). "Jam with Jamendo: Querying a Large Music Collection by Chords from a Learner’s Perspective". \emph{AM '18}. \emph{University of Wrexham}, Wrexham, UK.\\
\years{2018e} \textsc{Oral Presenter}. (July 12, 2018). “Audio Commons: Challenges and Opportunities of Using Online Repositories in Music Production and Performance". \emph{Filmuniversität Babelsberg Konrad Wolf}. Potsdam, Germany.\\
\years{2018d}\textsc{Panel Member}. (July 4, 2018) \emph{The Disturbing Discussion about Innovation} with Nicolas d'Alessandro (panelist), Tom Mitchell (panelist), Anna Xambó (panelist), and Matthias Strobel (moderator). Wallifornia MusicTech Hackathon. Liège, Belgium.\\
\years{2018c}\textsc{Panel Member}. (June 6, 2018) \emph{Future, Democratization, and Globalization of NIMEs} with Onyx Ashanti (panelist), Peter Nyboer (panelist), Anna Xambó (panelist), Pamela Z (panelist) and R. Benjamin Knapp (moderator). NIME '18. Moss Arts Center: Anne and Ellen Fife Theatre. Blacksburg, VA, USA.
\years{2018b} \textsc{Oral Presenter}. (June 6, 2018). “Who Are the Women Authors in NIME?---Improving Gender Balance in NIME Research". \emph{NIME '18}. Virginia Tech, Blacksburg, Virginia, USA.\\
\years{2018a} \textsc{Oral Presenter}. (April 21, 2018). “Live Repurposing of Crowdsourced Sounds: Challenges and Opportunities of Using Online Repositories in Music Performance". \emph{Sonorities Symposium, Sonorities Festival}. Queen's University Belfast, Belfast, Northern Ireland.\\
\years{2017} \textsc{Oral Presenter}. (August 24, 2017). “Turn-taking and Chatting in Collaborative Music Live Coding". \emph{AM '17}. London.\\
\years{2016b} \textsc{Oral Presenter}. (July 2, 2016). “Challenges and New Directions for Collaborative Live Coding in the Classroom". \emph{ICLI 2016}. University of Sussex, Brighton, UK.\\
\years{2016a} \textsc{Keynote Speaker}. (April 22, 2016). “Anna Xambó and Liz Dobson in Conversation". \emph{Women in Sound Women on Sound 2016: Educating girls in sound}, University of Lancaster. Lancaster, UK.\\
\years{2015} \textsc{Lightning Talk Speaker}. (August 14, 2015). “EarSketch: A STEAM Approach to Broadening Participation in Computer Science Principles". \emph{RESPECT 2015}. Charlotte, NC. USA.\\
\years{2014b} \textsc{Oral Presenter}. (July 1, 2014). “SoundXY4: Supporting Tabletop Collaboration and Awareness with Ambisonics Spatialisation". \emph{NIME '14}. Goldsmiths University, London.\\
\years{2014a} \textsc{Oral Presenter}. (April 30, 2014). “Let's Jam the Reactable: Peer Learning during Musical Improvisation with a Tabletop Tangible Interface". \emph{CHI '14}. Toronto, ON, Canada.\\
\years{2013} \textsc{Oral Presenter}. (November 11, 2013). “Tabletop Tangible Interfaces for Music Performance and Implications for Tabletop Research". \emph{School of Computing}, University of Kent. Kent, UK.\\
\years{2011b} \textsc{Oral Presenter}. (August 2, 2011). “Multi-touch Interaction Principles for Collaborative Real-time Music Activities: Towards a Pattern Language". \emph{ICMC '11}. University of Huddersfield. Huddersfield, UK.\\
\years{2011a} \textsc{Oral Presenter}. (July 4, 2011). “Collaborative Music Interaction on Tabletops: An HCI Approach". \emph{BCS HCI 2011 Workshop on When Words Fail: What can Music Interaction tell us about HCI?}. Newcastle Upon Tyne, UK.\\
\years{2010} \textsc{Oral Presenter}. (July 23, 2010). “Issues and Techniques for Collaborative Music Making on Multi-touch Surfaces". \emph{SMC '10}. Universitat Pompeu Fabra, Barcelona.\\
\years{2008c} \textsc{Panel Member} together with Alsina, A., Ferrete, J. and Roma, G. (October 31, 2008). “Freesound, Sons de Barcelona y Freesound Radio: Proyectos colaborativos alrededor del sonido" (Freesound, Sons de Barcelona \& Freesound Radio: Collaborative Projects around sound). \emph{IV Cicle de Converses d'Antropologia Sonora}, Institució Milá i Fontanals (CSIC). Barcelona.\\
\years{2008b} \textsc{Panel Member} together with Alsina, A., Ferrete, J. and Roma, G. (2008). “Freesound.org, Freesound Radio i Sons de Barcelona" (Freesound.org, Freesound Radio \& Sons de Barcelona"). \emph{Facultat de Belles Arts (Faculty of Fine Arts)}, Universitat de Barcelona. Barcelona.\\
\years{2008a} \textsc{Panel Member} together with Alsina, A., de Jong, B., Loscos, A. and Roma, G. (September 27, 2008). “Influencia de la tecnología en la evolución de la música y la industria" (Influence of the technology in the evolution of music and industry). \emph{NetAudio}, CCCB. Barcelona. \href{https://www.youtube.com/watch?v=6JlCCvYXrHY}{[video]}\\
\years{2007}\textsc{Oral Presenter} together with Roma, G. (September 20, 2007). “A Sound Editor with a Tangible Interface". \emph{SCSymposium(2007)}, DCM, The Hague, The Netherlands.

\subsection*{Own Institution}
\noindent

\years{2021} \textsc{Oral Presenter}. (November 10, 2021). “Music performance with crowdsourced sounds: collaboration by chance". \emph{MUST5001 Aesthetics and Ideas in the Sonic Arts}. DMU, Leicester, UK.\\
\years{2020} \textsc{Oral Presenter}. (November 25, 2020). “Music performance with crowdsourced sounds: collaboration by chance". \emph{MUST5001 Aesthetics and Ideas in the Sonic Arts}. DMU, Leicester, UK.\\
\years{2019c} \textsc{Oral Presenter} together with Jensenius, A., Jawad, K. and Aandahl, E. (May 23, 2019). “International joint Master's programme in Music, Communication \& Technology". \emph{Instituttseminar for sosiologi og statsvitenskap}. NTNU Department of Sociology and Political Science, Rockheim, Trondheim, Norway.\\
\years{2019b} \textsc{Oral Presenter}. (March 25, 2019). “Summary of my research and teaching 2018-2019". \emph{Faglig Forum}. Music Technology, NTNU, Trondheim, Norway.\\
\years{2019a} \textsc{Oral Presenter} together with Jawad, K. and Lesteberg, M. (January 31, 2019). “Women Nordic Music Technology". \emph{Girls Geek Dinner Trondheim}. Music Technology, NTNU, Trondheim, Norway.\\
\years{2018c} \textsc{Lightning Talk Speaker}. (September 28, 2018). “Challenges and Opportunities of Collaborative Music Live Coding: A Practitioner's Approach". \emph{The Raw and The Cooked, Inter/sections 2018}. Café 1001, London, UK.\\
\years{2018b} \textsc{Oral Presenter}. (August 13, 2018). “Women in Music Tech: A Case Study". \emph{Oppstartseminar (Institutt for musikk)}. Dokkhuset, Trondheim, Norway.\\
\years{2018a} \textsc{Oral Presenter}. (August 13, 2018). “A Journey Through My Research and Creative Practice". \emph{Oppstartseminar (Institutt for musikk)}. Dokkhuset, Trondheim, Norway.\\
\years{2017b} \textsc{Oral Presenter}. (June 19, 2017). “Computational Challenges of Co-creation in Collaborative Music Live Coding: An Outline". \emph{CCW2017: Co-Creation Workshop, ICC 2017}. Atlanta, GA, USA.\\
\years{2017a} \textsc{Panel Member} together with Ikkache, L. (May 4, 2017). “Women in Music Tech 2016--2017". Oral presentation and discussion. \emph{GTCMT}, Geogia Tech, Atlanta, GA, USA.\\
\years{2016d} \textsc{Lightning Talk Speaker}. (November 2, 2016). “Tangible User Interfaces and Tabletops". \emph{First Annual Women and Music Tech Concert and Reception}, The Garage, Atlanta, GA. USA.\\
\years{2016c} \textsc{Panel Member} together with Ikkache, L. and Jackson, D. (May 5, 2016). “Women in Sound." Oral presentation and discussion. \emph{GTCMT}, Geogia Tech, Atlanta, GA, USA.\\
\years{2016b} \textsc{Oral Presenter}. (February 25, 2016). “Algorithmic Composition: My Personal Journey". Oral presentation as a guest speaker in Jason Freeman's \emph{Computer Music Composition} class. GTCMT, Georgia Tech, Atlanta, GA, USA.\\
\years{2016a} \textsc{Oral Presenter}. (January 26, 2016). “EarSketch: Computational Music Remixing for All". Oral presentation as a guest speaker in Barbara Ericson's \emph{Educational Technology} class. College of Computing, Georgia Tech, Atlanta, GA, USA.\\
\years{2015c} \textsc{Oral Presenter}. (September 3, 2015). “Musical Tabletops: Challenges and Opportunities for Computer-Supported Collaborative Music and HCI". \emph{College of Architecture Research Forum}, Georgia Tech. Atlanta, GA, USA.\\ 
\years{2015b} \textsc{Oral Presenter}. (August 27, 2015). “Musical Tabletops: Challenges and Opportunities for Computer-Supported Collaborative Music and HCI". \emph{GVU Center Brown Bag Seminar Series}, Georgia Tech, Atlanta, GA, USA. \href{https://www.youtube.com/watch?v=VzOdtsq8YyE}{[video]}\\
\years{2015a} \textsc{Oral Presenter}. (August 24, 2015). “Musical Tabletops: Challenges and Opportunities for Computer-Supported Collaborative Music and HCI". \emph{GTCMT Seminar Series}, Georgia Tech, Atlanta, GA, USA.\\ 
\years{2014} \textsc{Oral Presenter}. (April 9, 2014). “Let's Jam the Reactable: Peer Learning During Musical Improvisation with a Tabletop Tangible Interface". \emph{London Knowledge Lab}, London.\\
\years{2013} \textsc{Oral Presenter}. (June 2, 2013). “Tabletop Groupware for Music Performance: Design and Evaluation". \emph{CRC PhD Student Conference 2013}, OU, Milton Keynes, UK.\\
\years{2012} \textsc{Oral Presenter}. (June 12, 2012). “Collaboration on Interactive Tabletops for Music Performance: An Exploratory Study". \emph{CRC PhD Student Conference 2012}, OU, Milton Keynes, UK.\\
\years{2011b} \textsc{Oral Presenter}. (June 16, 2011). “Tabletop Groupware for Music Performance: Design and Evaluation". \emph{CRC PhD Student Conference 2011}, OU, Milton Keynes, UK.\\
\years{2011a} \textsc{Oral Presenter}. (May 17, 2011). “Tabletop Groupware for Music Performance: Design and Evaluation". \emph{2011 Doctoral Workshops Conference}, OU, Milton Keynes, UK.\\
\years{2010b} \textsc{Oral Presenter}. (June 8, 2010). “Issues and Techniques for Collaborative Music Making on Multi-touch Surfaces". \emph{CRC PhD Student Conference 2010}, OU, Milton Keynes, UK.\\
\years{2010a} \textsc{Oral Presenter}. (May, 2010). “Issues and Techniques for Collaborative Music Making on Multi-touch Surfaces". \emph{Music Research Day}, Music Research Studio, OU, Milton Keynes, UK.

\section*{Poster Presentations, Demos \& Workshops}
%\noindent

\subsection*{Poster Presentations \& Demos}
\noindent

{\years{2021} \textsc{Poster Presenter (online)} (June 15, 2021). “Live Coding with the Cloud and a Virtual Agent". \emph{NIME '21}. 
\emph{NYU Shanghai}, Shanghai, China.\\
{\years{2020} \textsc{Poster Presenter (online)} (July 23, 2020). ``Performing Audiences: Composition Strategies for Network Music using Mobile Phones''. \emph{NIME '20}. Birmingham, UK.}\\
\years{2018b} \textsc{Poster Presenter} together with Roma, G. (June 7, 2018). “Live Repurposing of Sounds: MIR Explorations with Personal and Crowdsourced Databases". \emph{NIME '18}. Blacksburg, Virginia, USA. \\
\years{2018a} \textsc{Demo Presenter} together with Skach, S. (March 19, 2018). “Embodied Interactions with E-Textiles and the Internet of Sounds for Performing Arts". \emph{TEI '18}. Stockholm, Sweden. \\
\years{2017c} \textsc{Poster Presenter}. (June 22, 2017). Authors: Weisling, A. and Xambó, A. “Constructing a Conceptual Framework for Collaborative Audiovisual Performance". \emph{ICCC '17}. Atlanta, GA, USA.\\
\years{2017b} \textsc{Poster Presenter}. (June 22, 2017). Authors: Weisling, A., Xambó, A., Magerko, B., Roma, G., Jacob, M., Bhanu, N., and Freeman, J. “TuneTable: A Tangible Computational Music Installation for Informal Learning". \emph{ICCC '17}. Atlanta, GA, USA.\\
\years{2017a} \textsc{Poster \& Demo Presenter}. (March 21, 2017). “Experience and Ownership with a Tangible Computational Music Installation for Informal Learning". \emph{TEI '17}. Yokohama, Japan.\\
\years{2016b} \textsc{Poster \& Demo Presenter}. (August 11, 2016). “Learning to Code Through MIR". \emph{Late-Breaking Demo Session of ISMIR 2016}. New York.\\
\years{2016a} \textsc{Poster \& Demo Presenter} together with Roma, G. (July 2, 2016). “Do the Buzzer Shake". \emph{ICLI 2016}. Brighton, UK.\\
\years{2015} \textsc{Poster \& Demo Presenter} together with McKlin, T. (August 14, 2015). “EarSketch: A STEAM Approach to Broadening Participation in Computer Science Principles". \emph{RESPECT 2015}. Charlotte, NC. USA.\\
\years{2014} \textsc{Poster Presenter} together with Price, S. (April 29, 2014). “Towards an Integrated Methodological Framework for Understanding Embodiment in HCI". \emph{CHI '14}. Toronto, ON. \href{https://www.youtube.com/watch?v=pRDHuCUltwo}{[video]}\\
\years{2012} \textsc{Demo Presenter}. (January 10, 2012). "Tangible Additive Sound Synthesis (TASS)". \emph{Welcome to the French Embassy}, OU. Milton Keynes, UK.\\
\years{2011d} \textsc{Poster Presenter}. (June 17, 2011). “Designing and Evaluating Interactive Systems: Musical Tabletops for Collective Music Performance". \emph{CRC PhD Student Conference 2011}, OU. Milton Keynes, UK.\\
\years{2011c} \textsc{Poster \& Demo Presenter} together with Milne, A. J. (May 30, 2011). “Hex Player — A Virtual Musical Controller". \emph{NIME '11}. Oslo, Norway.\\
\years{2011b} \textsc{Poster Presenter}. (March 8, 2011). “Designing and Evaluating Interactive Systems: Musical Tabletops for Collective Music Performance". \emph{The Open University Poster Competition 2011}. Milton Keynes, UK.\\
\years{2011a} \textsc{Poster Presenter}. (January 25, 2011) “TOUCHtr4ck: Democratic Collaborative Music". \emph{TEI '11}. Funchal, Madeira.\\
\years{2010b} \textsc{Poster Presenter}. (December 21, 2010). Xambó, A., Laney, R., Dobbyn, C., Jordà, S. “Multi-touch Interaction Techniques for Collaborative Music Activities". \emph{DMRN+5: Digital Music Research Network One-day Workshop 2010}, Queen Mary, University of London. London.\\
\years{2010a} \textsc{Poster Presenter}. (June 8, 2010). “Issues and Techniques for Collaborative Music Making on Multi-touch Surfaces". \emph{CRC PhD Student Conference 2010}, OU. Milton Keynes, UK.\\
\years{2008b} \textsc{Poster Presenter}. (June 9--11, 2010). “Interfaces for Sketching Musical Compositions". \emph{SMC Summer School 2008}. Genoa, Italy.\\
\years{2008a} \textsc{Poster Presenter} together with Roma, G. (June 6, 2008). “A Tabletop Waveform Editor for Live Performance". \emph{NIME '08}. Genoa, Italy.\\


\subsection*{Workshops}
\noindent

\years{2021d}Pardue, L., Martínez, J., \textbf{Xambó, A.}, Cavdir, D., Almeida, I. and Bin, A. (June 14, 2021). “Actions We Can Take to Improve Diversity and Inclusivity at NIME". Online workshop. \emph{NIME '21}, Shanghai, China.\\
\years{2021c}\textbf{Xambó, A.}, Roig, S. (January 25/27/29, 2021). “Performing With a Virtual Agent: Machine Learning for Live Coding". Online workshop. \emph{Leicester Hackspace}, Leicester, UK.\\
\years{2021b}Pardue, L., Martínez, J. and \textbf{Xambó, A.} (January 14, 2021). “NIME Diversity Workshop". Online workshop.\\
\years{2021a}\textbf{Xambó, A.}, Roig, S. (January 11/13/15, 2021). “Performing With a Virtual Agent: Machine Learning for Live Coding". Online workshop. \emph{l'ull cec}, Barcelona, Spain.\\
\years{2020c}\textbf{Xambó, A.}, Roig, S. (December 7/9/11, 2020). “Performing With a Virtual Agent: Machine Learning for Live Coding". Online workshop. \emph{IKLECTIK}, London, UK.\\
\years{2020b}\textbf{Xambó, A.} (September 25, 2020). “Creative Audio Programming for the Web with Tone.js". \emph{Women Who Code} workshop series, School of Media Arts, Columbia College Chicago, IL, USA.\\
\years{2020a}Jensenius, A.R., McPherson, A., \textbf{Xambó, A.}, Martin, C., Armitage, J., Granieri, N., Fiebrink, R., Naveda, L. (July 21, 2020). “NIME Publication Ecosystem Workshop". \emph{NIME '20}. Birmingham, UK.\\
\years{2019c} \textbf{Xambó, A.} (June 20--21, 2019). “Creative Audio Programming for the Web". \emph{Filmuniversität Babelsberg Konrad Wolf}, Potsdam, Germany. Organized by MA Creative Technologies (Module 5 Audiovisual Application Design).\\
\years{2019b}Jensenius, A.R., McPherson, A., \textbf{Xambó, A.}, Overholt, D., Pellerin, G., Bukvic, I.I., Fiebrink, R., Schramm, R. (June 3, 2019). “Open Research Strategies and Tools in the NIME Community". \emph{NIME '19}. Centro cultural, UFRGS, Porto Alegre, Brazil.\\
\years{2019a} Allik, A., \textbf{Xambó, A.} (January 16, 2019). “Musical Networks of Live Coders". \emph{International Conference on Live Coding 2019}, Medialab Prado, Madrid, Spain. \\
\years{2018b} \textbf{Xambó, A.} (July 12--13, 2018). “Creative Audio Programming". \emph{Filmuniversität  Babelsberg Konrad Wolf}, Potsdam, Germany. Organized by MA Creative Technologies.\\
\years{2018a} Allik, A., \textbf{Xambó, A.} (April 7--8, 2018). “Collaborative Network Music". \emph{Rewire 2018}, The Hague, The Netherlands. Organized by Music Hackspace. Funded by Rewire.\\
\years{2017} \textbf{Xambó, A.} (October 14, 2017). “Huddersfield Girl Geeks: Audiovisual Creative Coding with P5.js". \emph{Kirklees Libraries}, Huddersfield, UK. Funded by Google.\\
\years{2013} \textbf{Xambó, A.} (May 2, 2013). “Introduction to SuperCollider". \emph{Music Computing Meeting}, OU. Milton Keynes, UK.\\
\years{2012} \textbf{Xambó, A.}; Roma, G. and Bovermann, T. (April 15, 2012). “Tangible Musical Interfaces with SuperCollider". \emph{SuperCollider Symposium 2012}, Goldsmiths, University of London. London.%\\[1.2cm]%hack

\subsection*{Webinars}
\noindent

\years{2016} \textbf{Xambó, A.} (October 28, 2016). “Debugging with EarSketch". GTCMT, Georgia Tech, Atlanta, GA, USA.%\\[3cm]%hack

\section*{Discography}
%\noindent

\subsection*{Solo Albums}
\noindent

\years{2018}Anna Xambó. \emph{H2RI} [FLAC/MP3 files]. Chicago (IL, USA): pan y rosas discos.\\
\years{2013}peterMann. \emph{On the Go} [promo CD \& FLAC/MP3 files]. Barcelona: Carpal Tunnel.\\
\years{2011}peterMann. \emph{init} [promo CD \& FLAC/MP3 files]. Barcelona: Carpal Tunnel.

\subsection*{Group Albums}
\noindent

\years{2021}Dirty Electronics Ensemble, Jon.Ogara, and Anna Xambó. \emph{Dirty Dialogues} (live album) [FLAC/MP3 files]. Chicago (IL, USA): pan y rosas discos.\\
\years{2019}Anna Weisling and Anna Xambó. \emph{Beacon} (EP) [FLAC/MP3 files]. Barcelona: Carpal Tunnel.\\
\years{1996}La Más Fina. \emph{Zande Phondex} [CD]. Barcelona: Apache Productions.\\
\years{1994}La Más Fina. \emph{Como quien dice la hoja iberia extrafina} [Cassette]. Barcelona: Self-released.\\
\years{1992}Sosa's Cáustica. \emph{Paraponera Clavata} [Cassette]. Barcelona: Murmur Town.

\subsection*{Participation in Compilations}
\noindent

\years{2020b}Anna Xambó. “Magnets" (4 min 21 sec) in \emph{Compassion Through Algorithms Vol. II} [WAV/MP3 files]. Sheffield: Light Entries. \\	
\years{2020a}Anna Xambó. “Poème Symphonique For Tape Metronome: Variation I" (5 min 38 sec) in \emph{In Unison} [WAV/MP3 files]. Brussels: i.u. \\	
\years{2019}Anna Xambó. “Kicks \& Cuts" (2 min 21 sec). in \emph{10th Anniversary Festival en Tiempo Real - WoNoMute Playlist} [WAV file].\\
\years{2018}peterMann. “n02-petermann" (11 min 10 sec). in \emph{Noiselets} [FLAC/MP3 files]. Barcelona: Carpal Tunnel.\\
\years{2016}peterMann. “Go wild y'all" (1 min). in \emph{Microtopies 2016} [MP3 files]. Barcelona: Gracia Territori Sonor.\\
\years{2015}peterMann. “ldnsktch01" (1 min). In \emph{Microtopies 2015} [MP3 files]. Barcelona: Gracia Territori Sonor.\\
\years{2010}peterMann. “init11" (3 min 29 sec). In \emph{Electronic music from Catalonia 2010} [CD]. Barcelona: Catalan! Arts / Sonar, Barcelona.

\subsection*{Broadcasting}
\noindent

\years{2021c}Dirty Electronics Ensemble, Jon.Ogara \& Anna Xambó's “Dirty Dialogues". (November 8, 2021). Framework radio \#776\\
\years{2021b}Anna Xambó's “Magnets". (April 12, 2021). Two Foot Left (Mixcloud playlist). Sheffield Live! Sheffield, UK.\\
\years{2021a}Anna Xambó's “Magnets". (February 2, 2021). TOPLAP \#2 w/ hangar.org (Ramon Casamajó). Audio Formal on Dublab.es. Barcelona, Spain.\\
\years{2020}Anna Xambó's “Magnets". (November 9, 2020). Sleepsang - Hyperobjects w/ Sleepsang (Mixcloud playlist). Sheffield, UK.\\
\years{2019}Anna Xambó's “Footsteps". Feminatronic \#161 - Your 2019 Playback and More (SoundCloud playlist).\\
\years{2018c}Anna Xambó's “H2RI.01-04". (June 21, 2018). Rare Frequency on WZBC 90.3 FM Newton Boston College Radio. Boston, MA, USA.\\
\years{2018b}Anna Xambó's “H2RI.011". (June 10, 2018). NILUCCIO ON NOISE. Podcast \#154 (May 2018).\\
\years{2018a}Anna Xambó's “H2RI.07". (May 17, 2018). No Pigeonholes EXP on KOWS-FM.\\
\years{2013f}peterMann's “og02". (July 28, 2013). BiP\_HOp Generation on Radio Grenouille.\\
\years{2013e}peterMann's “og01", og05, og07 \& og09. (June 23, 2013). Framework radio \#426.\\
\years{2013d}peterMann's “og01". (March 28, 2013). Rare Frequency on WZBC 90.3 FM Newton Boston College Radio. Boston, MA, USA.\\
\years{2013c}peterMann's “og01" \& “og10". (March 2, 2013). Onda Sonora.\\
\years{2013b}peterMann's selection of \emph{On The Go}'s tracks. (February 3, 2013). RNE Atmósfera. Madrid, Spain.\\
\years{2013a}peterMann's “og02". (February 2, 2013). Störung Radio 127 on ScannerFM. Barcelona, Spain.\\
\years{2010b}peterMann's “init 10--12". (December 18, 2010). Onda Sonora.\\
\years{2010a}peterMann's “init 2". (April 12, 2010). Sismógrafo.%\\[2cm]%hack

\section*{Selected Performances}
%\noindent

\subsection*{Solo Performances}
\noindent

\years{2022c}\textbf{Xambó, A.} (March 19, 2022). ``Make Noise Not War (live coding session)''. \emph{Algorave 10th Birthday Party}. Online event. Streaming from Sheffield, UK.\\
\years{2022b}\textbf{Xambó, A.} (March 11, 2022). ``detuning a tuning (live coding piece)''. \emph{BEAST @ Centrala: Anna Xambó Sedó, Milad K. Mardakheh}. Centrala, Birmingham, UK.\\
\years{2022a}\textbf{Xambó, A.} (March 8, 2022). ``Algonoise feat. Olympe de Gauges (live coding session)''. \emph{LIVECODERA a global live coding community gathering on International Women’s Day}. Online event. Streaming from Sheffield, UK.\\
\years{2021e}\textbf{Xambó, A.} (November 10, 2021). ``Magnets (live coding piece)''. \emph{Concert PACE 1 Live Music from MTI2. Works by Simon Emmerson, Anna Xambó Sedó, John Richards, and Leigh Landy}. Electroacoustic Music Studies 2021. Leicester, UK.\\
\years{2021d}\textbf{Xambó, A.} (July 6, 2021). ``Live Coding with Crowdsourced Sounds and A Virtual Agent Companion''. \emph{WAC 21}. Barcelona, Spain. Online event. Streaming from Sheffield, UK.\\
\years{2021c}\textbf{Xambó, A.} (June 17, 2021). ``A Live Coding Session With the Cloud and a Virtual Agent''. \emph{NIME 21}. Shanghai, China. Online event. Streaming from Sheffield, UK.\\
\years{2021b}\textbf{Xambó, A.} (May 6, 2021). ``Sonic Haikus''. \emph{ARTIFICIA Festival}. Online event. Barcelona, Spain.\\
\years{2021a}\textbf{Xambó, A.} (February 21, 2021). ``They Are the Robots: a live coding session''. \emph{Transnodal TOPLAP}. Online event. Streaming from Sheffield, UK.\\
\years{2020e}\textbf{Xambó, A.} (December 12, 2020). ``A live coding session''. \emph{Similar Sounds: A Virtual Agent in Live Coding}. IKLECTIK, London. Online event. Streaming from Sheffield, UK.\\
\years{2020d}\textbf{Xambó, A.} (November 14, 2020). ``A live coding session in binaural audio''. \emph{Sound Junction Satellites: Live Coding \& 3-D Sound -- Online Live Stream}. University of Sheffield Concerts/Algomech, Sheffield, UK. Online event. Streaming from Sheffield, UK.\\
\years{2020c}\textbf{Xambó, A.} (July 17, 2020). ``A Live Coding Exploration of How the Network Sounds''. \emph{Network Music Festival 2020}. Streaming from Sheffield, UK.\\
\years{2020b}\textbf{Xambó, A.} (March 22, 2020). ``Crowdsourced Eulerisms''. \emph{Eulerroom Equinox 2020}. Streaming from Sheffield, UK.\\
\years{2020a}\textbf{Xambó, A.} (February 28, 2020). ``A situated live coding session''. \emph{MTI Concert} celebrating 20 years of Music, Technology and Innovation at DMU with current and past staff and PhD alumni. PACE 1, Leicester, UK.\\
\years{2019}\textbf{Xambó, A.} (January 18, 2019). ``Live coding with crowdsourced sounds \& a drum machine''. \emph{International Conference on Live Coding}, Medialab Prado, Madrid, Spain.\\
\years{2018d}\textbf{Xambó, A.} (November 24, 2018). ``A session on participatory mobile music and live coding using crowdsourced sounds''. \emph{NTNU Research Concert}. Dokkhuset. Trondheim, Norway.\\
\years{2018c}\textbf{peterMann.} (September 28, 2018). ``Live coding session''. \emph{The Raw, Inter/sections 2018}. Café 1001. London, UK.\\
\years{2018b}\textbf{Xambó, A.} (September 19, 2018). Audience device participation piece. ``Imaginary Berlin''. \emph{WAC 18}. Factory Berlin. Berlin, Germany.\\  
\years{2018a}\textbf{Xambó, A.} (August 9, 2018). Live. ``MareNostrum''. \emph{Cube Fest}. Moss Arts Center. Blacksburg, VA, USA.\\
\years{2017}\textbf{peterMann}. (January 8, 2017). Live coding session. \emph{Noiselets: A Noise Music Microfestival}. Freedonia, Barcelona, Spain.\\
\years{2016b}\textbf{Xambó, A.} (April 22, 2016). Live coding with EarSketch. \emph{Women in Sound Women on Sound 2016: Educating girls in sound}. Jack Hylton Music Room, University of Lancaster. Lancaster, UK.\\
\years{2016a}\textbf{peterMann}. (April 22, 2016). Live. \emph{Women in Sound Women on Sound 2016: Educating girls in sound}. Jack Hylton Music Room, University of Lancaster. Lancaster, UK.\\
\years{2013}\textbf{Xambó, A.} (October 4, 2013). Live coding session. \emph{Perspectives on Multichannel Live Coding}. PHONOS. Sala Polivalent, UPF. Barcelona, Spain.\\
\years{2012}\textbf{peterMann.} (September 20, 2012). Live. \emph{Crispy Crunchy Creaky}. Niu. Barcelona, Spain.\\
\years{2006}\textbf{peterMann.} (June 10, 2006). Live. \emph{5a Mostra Sonora i Visual | Convent Sant Agustí}. Barcelona, Spain.

\subsection*{Collaborative Performances}
\noindent
\years{2021b}\textbf{Xambó, A.}, Goudarzi, V. (November 21, 2021). “immerse in the lake" (online performance). \emph{Jefferson Park EXP}. Chicago, IL, USA. Streaming from Sheffield, UK.\\
\years{2021a}Goudarzi, V., \textbf{Xambó, A.} (September 24, 2021). “Livesourcing: Audience Participation in a Live Coding Performance" (online performance). \emph{Ear Taxi Festival}. Chicago, IL, USA. Streaming from Sheffield, UK.\\
\years{2020}Schroeder, F., Meireles, M., Mannone, M., Papadomanolaki, M., Brown, N., Stolfi, A., Alarcón, X., \textbf{Xambó, A.} (July 28, 2020). “Absurdity" (telematic performance by the Female Laptop Orchestra (FLO)). \emph{Physically Distant \#2: More Online Talks on Telematic Performance}. Streaming from Sheffield, UK.\\
\years{2019}Brandtsegg, Ø., \textbf{Xambó, A.}, Engum, T., Bergsland, A., Waadeland, C.H. (December 4, 2019). “Trondheim EMP Repository processing". \emph{WAC 2019}. Rockheim, Trondheim, Norway.\\
\years{2018e}Brown, N., Chudy, M., Dobson, L., Hoel, A.M., Ikkache, L., Pase, T., Schroeder, F., Stolfi, A., Wilkie, S., \textbf{Xambó, A.} (November 22, 2018). “Transmusicking II". Sonic Arts Research Center, Queen's University Belfast. Belfast, Northern Ireland.\\
\years{2018d}Martin, C.M., Lesteberg, M., Jawad, K., Aandahl, E., \textbf{Xambó, A.} (August 29, 2018). ``Stillness during Tension''. \emph{MCT Open Seminar}. MCT Portal, NTNU / UiO, Trondheim / Oslo.\\   
\years{2018c}Martin, C.M., \textbf{Xambó, A.}, Visi, F., Morreale, F., Jensenius, A.R. (June 5, 2018). ``Stillness during Tension''. \emph{NIME '18}. Moss Arts Center: Anne and Ellen Fife Theatre. Blacksburg, VA, USA.\\   
\years{2018b}Weisling, A., \textbf{Xambó, A.} (June 5, 2018). “Beckon". \emph{NIME '18}. Moss Arts Center: Anne and Ellen Fife Theatre. Blacksburg, VA, USA.\\
\years{2018a}Weisling, A., \textbf{Xambó, A.} (March 20, 2018). “Beacon". \emph{TEI '18}. Kulturhuset. Stockholm, Sweden.\\
\years{2017d}Brown, N., Chudy, M., Papadomanolaki, M., Wilkie, S., Pase, T., Stolfi, A., Schroeder, F., \textbf{Xambó, A.}, Ikkache, L., Freeman, J., Ganesh, S., Kerure, A., Narang, J., Tsuchiya, T. (August 25, 2017). “Transmusicking I". \emph{AM '17}. Oxford House Theatre. London, UK. \\
\years{2017c}\textbf{Xambó, A.}, Roma, G. (August 21, 2017). “Hyperconnected Action Painting". \emph{WAC 2017}. Oxford House Theatre. London, UK. \\
\years{2017b}Weisling, A., \textbf{Xambó, A.} (May 16, 2017). “Beacon". \emph{NIME 2017}. Stengade. Copenhagen, Denmark.\\
\years{2017a}Weisling, A., \textbf{Xambó, A.} (February 11, 2017). “Beacon". \emph{Root Signals Festival 2017}. Georgia Southern University. Statesboro, Georgia, United States.\\
\years{2016b} Roma, G., \textbf{Xambó, A.}, Freeman, J. (November 2, 2016). Do the Buzzer Shake. \emph{The First Annual Women in Music Tech: Concert and Reception}. The Garage. Atlanta, GA, USA.\\
\years{2016a} Roma, G., \textbf{Xambó, A.}, Freeman, J. (July 1, 2016). “Do the Buzzer Shake". \emph{ICLI 2016}. St Mary's Church, Kemptown, Brighton, UK.\\
\years{2012}pulso (Roma, G., \textbf{Xambó, A.}). (March 15, 2012). Live coding session. \emph{Live Coding Sessions}. Niu. Barcelona, Spain.\\
\years{2004}pulso (Roma, G., \textbf{Xambó, A.}). (May 29, 2004). Live. \emph{Minima Festival}. Gandía, Spain.\\
\years{2002}b4ng (Roma, G., \textbf{Xambó, A.}, Brugos, Clarens). (June 13, 2002). Live. \emph{Sonar Festival}. Barcelona, Spain. 

\section*{Mastering (other's work)}
\noindent

\years{2018}\emph{Noiselets} [FLAC/MP3 files]. Barcelona: Carpal Tunnel.

\section*{Other Creative Products}
%\noindent

\subsection*{Awarded Music Hacks}
\noindent

\years{2014}“crowdj". \emph{Music Hack Day}. Barcelona, Spain.\\ 
Prize: Rdio prize.\\
Role: Concept, part of the implementation and user interface design.\\
Collaborator: Gerard Roma.\\
\years{2012b}“Soundscape Turntablism". \emph{Music Hack Day}. Barcelona, Spain.\\ 
Prize: Reactable prize, Zvooq prize.\\  
Role: Concept, part of the implementation and tangible user interface design.\\
Collaborator: Gerard Roma.\\
\years{2012a}“Soundscape DJ". \emph{Music Tech Fest}. London, UK.\\ 
Prize: Warp Records prize. \\
Role: Concept, part of the implementation and tangible user interface design.\\
Collaborator: Gerard Roma.

\subsection*{Code}
\noindent

%\years{2022--present}personic.\\
%Role: Concept and implementation.\\
\years{2020--present}MIRLCa: \href{https://github.com/mirlca/code}{github.com/mirlca/code}.\\
Role: Concept and implementation.\\
\years{2019--present}MIRLC 2.0.\\
Role: Concept and implementation.\\
\years{2018b}Embedded AudioCommons: \href{https://github.com/AudioCommons/embedded-audiocommons}{github.com/AudioCommons/embedded-audiocommons}.\\
Role: Concept and implementation.\\
\years{2018a}HCI Python Utils: \href{https://github.com/axambo/hci-python-utils}{github.com/axambo/hci-python-utils}.\\
Role: Concept and implementation.\\
\years{2017--present} WACastMix: \href{http://annaxambo.me/code/WACastMix}{annaxambo.me/code/WACastMix}.\\
Role: Concept and implementation.\\
\years{2016--2019}MIRLC: \href{https://github.com/axambo/MIRLC}{github.com/axambo/MIRLC}.\\
Role: Concept and implementation.\\
\years{2017b} HAP: \href{https://github.com/axambo/HAP}{github.com/axambo/HAP}.\\
Role: Concept and implementation.\\
\years{2017a}Beacon: \href{https://github.com/axambo/beacon}{github.com/axambo/beacon}.\\
Role: Concept and implementation of the audio engine.\\
\years{2016}Algonoise.: \href{https://github.com/axambo/algonoise}{github.com/axambo/algonoise}.\\
Role: Concept and implementation.\\
\years{2014}SoundXY4: The Art of Noise: \href{https://github.com/axambo/soundxy4}{github.com/axambo/soundxy4}.\\
Role: Concept, implementation and tangible user interface design.\\
\years{2012}SoundXY: \href{https://github.com/axambo/soundxy2}{github.com/axambo/soundxy2}.\\
Role: Concept, implementation and tangible user interface design.

\subsection*{Video Creations \& Animation Films}
\noindent

\years{2003}Xambó, A. \emph{Cosmogonias} (3 min). Spain. Video creation | Animation film.\\
\years{2002b}Xambó, A. \emph{b.scope} (3 min). Spain. Video creation.\\ %3:23
\years{2002a}Xambó, A. \emph{Transdata Pr.} (5 min). Spain. Video creation.\\
\years{2000}Xambó, A.  \emph{clubsfera} (3 min). Spain. Video creation | Animation film.\\
\years{1999}Xambó, A. \emph{Mitösöma} (10 min). Spain. Video creation | Animation film.\\
\years{1998c}Xambó, A. \emph{Lufthansa} (3 min). Spain. Videoclip for La Más Fina.\\
\years{1998b}Xambó, A. \emph{Neila} (2 min). Spain. Video creation.\\
\years{1998a}Xambó, A. \emph{Sueños} (1 min). Spain. Video creation | Animation film.

\subsection*{Installations \& Visuals}
\noindent

\years{09/2002}\emph{I love Japan}, Fake Industries, Circuit Festival, Barcelona.\\
%Description: Audiovisual installation for Divinas Palabras' Japan collection. \\
Role: Visuals.\\
Collaborators: Urtzi Grau (director), Emma Dünner, Jorge Meneses, Ana Otero.\\

\years{03/2002--08/2002}\emph{Astoria (cinema \& restaurant)}, Barcelona. \\
%Description: Visuals for the opening and first season of the Astoria (cinema \& restaurant) venue.\\
Role: Co-filming and visuals.\\ 
Collaborators: Babylon Cannes (concept).\\

\years{09/2001}\emph{Eme3density, Second Architectural Market}, Centre de Cultura Contemporània de Barcelona (CCCB), Barcelona.\\
Role: Visuals \& Flash programming. \\ %(Flash animation and programming of the website presented in the event).\\
Collaborators: Urtzi Grau (curator), Ana Otero (artistic director).

%\hrule
\section*{Teaching \& Coordination}
%\noindent

\subsection*{Graduate Courses}
\noindent

\years{10/2019}Course: \emph{MCT4000 Music Communication and Technology)} \# Students: $\sim$14. \\
Master of Music, Communication and Technology (MCT), Norwegian University of Science and Technology (NTNU), Trondheim, Norway.\\ 
Role: Coordination of a 15-credit course which includes 7 modules.\\
\years{10/2019}Course: \emph{MCT4000 Human-Computer Interaction Lectures (2nd edition)} (8 h). \# Students: $\sim$14. \\
Master of Music, Communication and Technology (MCT), Norwegian University of Science and Technology (NTNU), Trondheim, Norway.\\ 
Role: Creation of syllabus, creation of content, instruction and assessment.\\
\years{10/2019}Course: \emph{MCT4000 Physical Computing Workshop (2nd edition)} (28 h). \# Students: $\sim$14. \\
Master of Music, Communication and Technology (MCT), Norwegian University of Science and Technology (NTNU), Trondheim, Norway.\\ 
Role: Creation of syllabus, creation of content, instruction and assessment.\\
\years{1/2019}Course: \emph{MCT4046 Sonification and Sound Design} (58 h). \# Students: $\sim$10. \\
Master of Music, Communication and Technology (MCT), Norwegian University of Science and Technology (NTNU), Trondheim, Norway.\\ 
Role: Creation of syllabus, creation of content, instruction, coordination of guest lectures, and assessment.\\
\years{1/2019}Course: \emph{MCT4048 Audio Programming} (58 h). \# Students: $\sim$11. \\
Master of Music, Communication and Technology (MCT), Norwegian University of Science and Technology (NTNU), Trondheim, Norway.\\ 
Role: Creation of syllabus, creation of content, instruction and assessment.\\
\years{10/2018}Course: \emph{MCT4000 Human-Computer Interaction Lectures} (8 h). \# Students: $\sim$15. \\
Master of Music, Communication and Technology (MCT), Norwegian University of Science and Technology (NTNU), Trondheim, Norway.\\ 
Role: Creation of syllabus, creation of content, instruction and assessment.\\
\years{10/2018}Course: \emph{MCT4000 Physical Computing Workshop} (28 h). \# Students: $\sim$15. \\
Master of Music, Communication and Technology (MCT), Norwegian University of Science and Technology (NTNU), Trondheim, Norway.\\ 
Role: Creation of syllabus, creation of content, instruction and assessment.\\


\subsection*{Undergraduate Courses}
\noindent

\years{01/2022--04/2022}Course: \emph{MATD3039 Advanced Musical Electronics} (30 h). \# Students: $\sim$4. \\
De Montfort University, Leicester, UK.\\ 
\years{10/2021--12/2021}Course: \emph{MATD3009 Advanced Digital Signal Processing Music} (30 h). \# Students: $\sim$4. \\
De Montfort University, Leicester, UK.\\ 
\years{10/2021--12/2021}Course: \emph{MATD2004 Further Digital Signal Processing} (30 h). \# Students: $\sim$13. \\
De Montfort University, Leicester, UK.\\ 
\years{01/2020--04/2021}Course: \emph{MATD1019 Audio Electronics Fundamentals} (12 h). \# Students: $\sim$10. \\
De Montfort University, Leicester, UK.\\ 
Role: Course leader, creation of content for lectures, instruction and assessment.\\
\years{01/2021--04/2021}Course: \emph{TECH3011 Studio Technology: Audio Electronics} (36 h). \# Students: $\sim$16. \\
De Montfort University, Leicester, UK.\\ 
Role: Creation of content for lectures and labs, instruction and assessment.\\
\years{10/2020--12/2020}Course: \emph{MATD2004 Further Digital Signal Processing} (30 h). \# Students: $\sim$14. \\
De Montfort University, Leicester, UK.\\ 
Role: Course leader, creation of content for lectures, seminars and labs (blended teaching), instruction and assessment.\\
\years{10/2020--12/2020}Course: \emph{MATD2019 Further Audio Electronics} (10 h). \# Students: $\sim$10. \\
De Montfort University, Leicester, UK.\\ 
Role: Course leader, creation of content for lectures and seminars (blended teaching), instruction and assessment.\\
\years{01/2020--04/2020}Course: \emph{MATD1019 Audio Electronics Fundamentals} (12 h). \# Students: $\sim$12. \\
De Montfort University, Leicester, UK.\\ 
Role: Course leader, creation of content for lectures, instruction and assessment.\\
\years{01/2020--04/2020}Course: \emph{TECH2019 Audio Technology 2: Audio Electronics} (12 h). \# Students: $\sim$16. \\
De Montfort University, Leicester, UK.\\ 
Role: Course leader, creation of content for lectures, instruction and assessment.\\
\years{01/2020--04/2020}Course: \emph{TECH3011 Studio Technology: Audio Electronics} (36 h). \# Students: $\sim$16. \\
De Montfort University, Leicester, UK.\\ 
Role: Creation of content for lectures and labs, instruction and assessment.\\
\years{02/2004--06/2004}Course: \emph{Experimental Motion Graphics} (45 h). \# Students: $\sim$15. \\
Centre de la Imatge i la Technologia Multimèdia, Universitat Politècnica de Catalunya, Terrassa, Barcelona.\\ 
Role: Co-creation of syllabus, creation of content, instruction and assessment.\\
\years{10/2003--02/2004}Course: \emph{Crossmedia} (45 h). \# Students: $\sim$15. \\
BAU Escola de Disseny, Universitat de Vic, Barcelona.\\ 
Role: Co-creation of syllabus, creation of content, instruction and assessment.\\
\years{11/2003--06/2004}
Course: \emph{Digital Compositing with Adobe AfterEffects} (45 h). \# Students: $\sim$10. \\ 
Media Art Institute Fak d'Art, Barcelona.\\ 
Role: Creation of syllabus, creation of content, instruction and assessment.\\
\years{11/2003--06/2004}
Course: \emph{Photography in Motion} (45 h). \# Students: $\sim$10. \\ 
Media Art Institute Fak d'Art, Barcelona.\\ 
Role: Creation of syllabus, creation of content, instruction and assessment.\\
\years{11/2003--06/2004}
Course: \emph{Type in Motion} (45 h). \# Students: $\sim$10.\\
Media Art Institute Fak d'Art, Barcelona.\\ 
Role: Creation of syllabus, creation of content, instruction and assessment.\\
\years{11/1999--06/2003}
Course: \emph{Computer Animation} (90 h). \# Students: $\sim$15. \\ 
Media Art Institute Fak d'Art, Barcelona.\\ 
Role: Creation of syllabus, creation of content, instruction and assessment.

\subsection*{Professional Courses}
\noindent

\years{04/2004--05/2005}
Course: \emph{Usability} (12 h). \# Students: $\sim$5.\\ 
Crea Formación, Barcelona.\\
Role: Instruction.\\
\years{04/2004--05/2005}
Course: \emph{Internet Design Techniques} (12 h). \# Students: $\sim$5.\\ 
Crea Formación, Barcelona.\\
Role: Instruction.\\
\years{04/2004--05/2005}
Course: \emph{Web Design with DreamWeaver} (24 h). \# Students: $\sim$5.\\
Crea Formación, Barcelona.\\
Role: Instruction.\\ 
\years{04/2004--05/2005}
Course: \emph{Multimedia Content with Adobe Flash} (16 h). \# Students: $\sim$5. \\
Crea Formación, Barcelona.\\
Role: Instruction.\\
\years{04/2004--05/2005}
Course: \emph{Flash Programming} (20 h) \# Students: $\sim$5.\\ 
Crea Formación, Barcelona.\\
Role: Instruction.\\
\years{04/2004--05/2005}
Course: \emph{Theoretical Aspects in Graphic Design} (12 h). \# Students: $\sim$5.\\ 
Crea Formación, Barcelona.\\
Role: Instruction.\\
\years{04/2004--05/2005}
Course: \emph{Video Edition with Adobe Premiere} (60 h) \# Students: 1.\\
Crea Formación, Barcelona.\\
Role: Creation of syllabus, creation of content and instruction.

\subsection*{Preschool \& Primary School Courses}
\noindent

\years{03/2004--06/2004}
Course: \emph{Crossmedia infantil} (11 h). \# Students (6--7 years old): $\sim$8.\\
Escola Magòria, Barcelona.\\
Role: Co-creation of syllabus, creation of content, instruction and assessment.\\
\years{03/2004--05/2004}
Course: \emph{Crossmedia infantil} (9 h). \# Students (9--10 years old): $\sim$15.\\
Escola Costa i Llobera, Barcelona.\\ 
Role: Co-creation of syllabus, creation of content, instruction and assessment.\\
\years{03/2004--05/2004}
Course: \emph{Crossmedia infantil} (12 h). \# Students (3--4 years old): $\sim$8.\\
Escola Glòries, Barcelona.\\
Role: Co-creation of syllabus, creation of content, instruction and assessment.%\\ [1.3cm]%hack

\section*{Supervision}
%\noindent

\subsection*{PhD Students}
\years{04/2022--present}2nd supervisor of Eva Kára. DMU.

\subsection*{Master Students}
\years{07/2021--10/2021}Creative portfolio \& short dissertation (master thesis) co-supervisor of Harri Bettsworth. Master thesis title: \emph{``Exploring How Uncanny Sound Can Trigger Fear In Video Game''}. Master programme: Music Technology and Innovation, DMU. \\
\years{01/2020--07/2020}Master thesis supervisor of Karolina Jawad. Master thesis title: \emph{``Gatekeepers by Design? Gender HCI for Audio and Music Hardware''}. Master programme: Music Communication Technology, NTNU. Our publication at ICLI '20 (see Peer-Reviewed Conference Papers) and the work at WoNoMute has inspired this work.\\
\years{01/2018--08/2018}Master thesis co-advisor of Tayjo Padmini Vaduru. Master thesis title: \emph{``Moodscape Generator: Automated Generation of Soundscapes''}. Computer Science, QMUL.\\
\years{09/2015--05/2016}Co-advisor of Marc Huet and Travis Gasque (master's students in Digital Media, School of Literature, Media, and Communication) and Anna Weisling (PhD student in Digital Media, School of LMC) for their graduate design project TuneTable. This work has been part of Brian Magerko’s Digital Media studio course at Georgia Tech. From this work we have published at TEI '17 (see Peer-Reviewed Conference Papers) and we have informed a successful and competitive NSF-funded grant (Advancing Informal STEM Learning Grant).\\
\years{09/2015--05/2017}Co-advisor of Pratik Shah (master student in Human-Centered Computing, School of Interactive Computing) with the research and design on adding collaborative features to EarSketch, an online platform for learning code by making music. This work has been part of the design and development of the NSF-funded project EarSketch, led by Jason Freeman. From this work we have published at the conferences ICLI '16 and AM '17, and also at the AES journal in 2018 (see Peer-Reviewed Conference Papers).

\subsection*{Undergraduate Students}
\years{09/2021--06/2022}Ongoing supervision of two L6 undergraduate students in music technology for their MATT3000 Research Project: Abbey Young working on ``A Practice-based Analysis of Practical Workshops to Improve Gender Equality in Music Technology Education in the UK'' and Tolu Ikuyinminu working on ``Investigating the Positive and Negative Impacts of the `UK Drill' Genre on Generation Z''. 

\section*{Mentoring}
\years{08/2018--12/2019}Mentor and advisor of the members of the organization \emph{WoNoMute}, including the co-chair and research assistant Karolina Jawad, and the research assistants Mari Lesteberg and Ane Bjerkan.\\
\years{05/2016--12/2017}Mentor and advisor of the members of the student-led organization \emph{Women in Music Tech}, including the chair of the organization, Léa Ikkache, the editor-in-chief of the newsletter Amruta Vidwans, and other organization members, such as Jyoti Narang.
 
\section*{Assessment}
\noindent

\years{02/2022}PhD review for Robin Foster, PhD programme: DMU. Leicester, UK.\\
\years{09/2021}External PhD examiner for Joaquín Roberto Díaz Durán. PhD thesis title: \emph{"Interfaz Cíborg: Dispositivo interactivo para reflexionar la relación arte, tecnología y cuerpo en la performance"} (\emph{Cyborg interface: Interactive device to reflect on the relationship between art, technology and the body in performance}). PhD programme: Investigación en Humanidades, Artes y Educación (\emph{Research in Humanities, Arts and Education}), Facultad de Bellas Artes de Cuenca, Universidad de Castilla - La Mancha, Spain.\\
\years{09/2021}External PhD examiner for Jack Armitage. PhD programme: Media \& Arts Technologies, Queen Mary University of London, UK. PhD thesis title: \emph{"Subtlety and Detail in Digital Musical Instrument Design"}.\\
\years{02/2021}PhD review (midway assessment) for Sam Topley, PhD programme: DMU/Midlands4Cities. Leicester, UK.\\
\years{05/2020}External assessor, IMV PhD Midway Assessment for Qichao Lan, Department of Musicology, Faculty of Humanities, University of Oslo, Norway.\\
\years{08/2019}External MSc examiner for Torgrim Rudland Næss, Master thesis title: \emph{"A Physical Intelligent Instrument Using Recurrent Neural Networks"}. Department of Informatics, Faculty of Mathematics and Natural Sciences, University of Oslo, Norway.\\
\years{11/2018--06/2019}Appraisal Committee Chair for Hilmar Thordarson, PhD thesis title: \emph{"Condis–conducting Digital System Extended Role Of The Conductor in Mixed Music Performance"}. Norwegian Programme for Artistic PhD Research, Norwegian Science and Technology University, Trondheim, Norway. 

\section*{Additional Experience}
%\noindent

\subsection*{Concerts Co-Organization}
\noindent

\years{2017}“Noiselets: A Noise Music Microfestival". (January 8, 2017). Freedonia, Barcelona.\\
\years{2016c}“The First Annual Women in Music Tech: Concert and Reception". (November 2, 2016). The Garage. Atlanta, GA, USA.\\
\years{2016b}“Audience device participation". (April 5, 2016). \emph{Web Audio Conference 2016}, Georgia Tech. Atlanta, GA, USA.\\
\years{2016a}“Live coding and the audiovisual web". (April 4, 2016). \emph{Web Audio Conference 2016}, Georgia Tech. Atlanta, GA, USA.\\
\years{2013b}“Perspectives on multichannel live coding". (October 4, 2013). PHONOS. Sala Polivalent, UPF. Barcelona.\\
\years{2013a}“Live Coding Sessions II". (March 22, 2013). Niu. Barcelona.\\
\years{2012}“Live Coding Sessions". (March 15, 2012). Niu. Barcelona.

\subsection*{Blogging}
\noindent

\years{06/2020--2021}\href{https://mirlca.dmu.ac.uk}{MIRLCAuto}, the website and blog of the research project \emph{MIRLCAuto: A Virtual Agent for Music Information Retrieval in Live Coding}. Creator and Author.\\ 
\years{09/2018--12/2019}\href{http://wonomute.no}{Women Nordic Music Technology}, the website and blog of the WoNoMute organization. Co-Creator and Coordinator.\\ 
\years{08/2018--12/2019}\href{https://mct-master.github.io}{MCT master blog}, the blog of the MCT master. Co-Creator and Co-Coordinator.\\ 
\years{05/2017--07/2018}\href{https://www.audiocommons.org}{Audio Commons}, the blog of the EU-funded project Audio Commons. Coordinator, Reviewer and Author.\\ 
\years{10/2016--present}\href{http://annaxambo.me/blog}{Anna Xambó's Blog}, the blog of my personal website. Creator and Author.\\ 
\years{05/2016--12/2017}\href{http://womeninmusictech.gatech.edu}{Women in Music Tech}, the newsletter of the Women in Music Tech organization. Co-Creator, Coordinator, Reviewer and Author.\\ 
\years{09/2013--08/2014}\href{http://midassblog.wordpress.com}{MIDAS's Blog}, the research blog of the MIDAS project. Co-Creator, Coordinator and Author. \\ 
\years{01/2010--12/2011} \href{http://postwimp.com}{postWIMP}, a blog on HCI and interaction design. Co-Creator, Coordinator and Author.\\
\years{03/2006--03/2009}\href{http://streetypes.blogspot.com}{streeTypes}, a blog on typography in public spaces. Creator and Author. 

\subsection*{Artistic Collective Projects}
\noindent

\years{2002}Co-Founder and Member of b4ng, a multidisciplinary collective in search of new forms of audiovisual communication. Barcelona.\\
\years{1998--2000}Co-Founder and Member of the experimental video collective jesus13. Barcelona. %\\ %[1.3cm]%hack

%\hrule
\section*{Professional Activities}
%\noindent

\subsection*{Professional Organization Member}
\noindent
\emph{Music, Technology and Innovation - Institute for Sonic Creativity (MTI${^2})$.}\\
\emph{Association for Computing Machinery (ACM)}.\\
\emph{International Computer Music Association (ICMA)}.\\
\emph{Intersections, Feminism, Technology \& Digital Humanities network (IFTe).}


\subsection*{Academic Programme Leader}
\noindent

\years{09/2020--present}\textsc{BSc Digital Music Technology Programme Leader}, Leicester Media School, DMU.\\
\years{01/2019--12/2019}\textsc{MCT Programme Study Leader}, Department of Music, NTNU.\\
\years{01/2019--12/2019}\textsc{MCT Programme Council Leader}, Department of Music, NTNU/UiO.

\subsection*{Organisation Founder / Chair}
\noindent

\years{2018-2019}\textsc{Co-Founder \& Chair}. \emph{Women Nordic Music Technology}, NTNU/UiO. Trondheim, Norway.\\
\years{2016--2017}\textsc{Co-Founder \& Co-Chair}. \emph{Women in Music Tech}, GTCMT, Georgia Tech. Atlanta, GA, USA.


\subsection*{Fellow}
\noindent

\years{2019--present}\emph{Associate Fellow of the Higher Education Academy}, United Kingdom.\\
\years{2018--2019}\emph{Visiting Lecturer}, Centre for Digital Music (C4DM), School of EECS, QMUL.


\subsection*{Officer / Consultant}
\noindent

\years{2020}\textsc{Diversity consultant}. \emph{2nd Conference on AI Music Creativity (MuMe + CSMC)}. Graz, Austria.\\
\years{2019--present}\textsc{WiNIME Officer}. \emph{International Conference of New Interfaces for Musical Expression}.


\subsection*{Conference Academic Chair / Meta-Reviewer / Local Chair}
\noindent

\years{2022}\textsc{Conference Programme Committee Member} (Meta-Reviewer). \emph{New Interfaces for Musical Expression} (2022). Auckland, New Zealand.\\
\years{2021b}\textsc{Conference Programme Committee Member} (Associate Chair). \emph{ACM Tangible, Embedded and Embodied Interaction 2022}. KAIST, Daejeon, Republic of Korea.\\
\years{2021a}\textsc{Conference Programme Committee Member} (Meta-Reviewer). \emph{New Interfaces for Musical Expression} (2021). Shanghai, China.\\
\years{2020e}\textsc{Conference Programme Committee Member} (Associate Chair). \emph{ACM Creativity \& Cognition 2021}. Venice, Italy.\\
\years{2020d}\textsc{Conference Programme Committee Member} (Associate Chair). \emph{ACM Tangible, Embedded and Embodied Interaction 2021}. Salzburg, Austria.\\
\years{2020c}\textsc{Conference Session Chair} (``Hardware and Software for SMC''). \emph{Sound and Music Computing Conference}. Torino, Italy.\\
\years{2020b}\textsc{Conference Poster Co-Chair}. \emph{The 1st International Workshop on the Internet of Sounds}, 27th FRUCT Conference, University of Trento, Italy.\\ 
\years{2020a}\textsc{Conference Programme Committee Member} (Meta-Reviewer). \emph{New Interfaces for Musical Expression} (2020). Birmingham, UK.\\
\years{2019g}\textsc{Conference Local Committee Member}. \emph{International Conference on Live Interfaces 2020}. Trondheim, Norway.\\ 
\years{2019f}\textsc{Conference General Co-Chair}. \emph{Web Audio Conference 2019}. Trondheim, Norway.\\ 
\years{2019e}\textsc{Conference Programme Committee Member} (Meta-Reviewer). \emph{Web Audio Conference 2019}. Trondheim, Norway.\\ 
\years{2019d}\textsc{Conference Session Chair} (``Mapping \& Sound Generation''). \emph{New Interfaces for Musical Expression 2019}. Porto Alegre, Brazil.\\ 
\years{2019c}\textsc{Conference Paper Co-Chair}. \emph{New Interfaces for Musical Expression 2019}. Porto Alegre, Brazil.\\ 
\years{2019b}\textsc{Conference Programme Committee Member} (Associate Chair). \emph{ACM Tangible, Embedded and Embodied Interaction 2020}. Sydney, Australia.\\ 
\years{2019a}\textsc{Conference Programme Committee Member} (Associate Chair). \emph{ACM Creativity \& Cognition 2019}. San Diego, USA.\\ 
\years{2018a}\textsc{Conference Programme Committee Member} (Associate Chair). \emph{ACM Spatial User Interaction 2018}. Berlin, Germany.\\ 
\years{2017d}\textsc{Conference Session Chair}. \emph{Web Audio Conference 2017}. London.\\
\years{2017a}\textsc{Conference Local Committee Member}. \emph{International Conference on Computational Creativity 2017}, Georgia Tech. Atlanta, GA, USA.\\
\years{2016}\textsc{Conference Music/Artworks Co-Chair}. \emph{Web Audio Conference 2016}, Georgia Tech. Atlanta, GA, USA.\\
\years{2011b}\textsc{Conference Session Chair} (``Laptop/Coding/NI''). \emph{International Computer Music Conference}. Huddersfield, UK.\\
\years{2011a}\textsc{Conference Committee Member}. \emph{CRC PhD Student Conference 2011}, OU. Milton Keynes, UK.

\subsection*{Conference Programme Committee Member (Reviewer)}
\noindent

\years{2020b}\emph{Workshop on the Internet of Sounds} (2020).\\ 
\years{2020a}\emph{Network Music Festival} (2020).\\
\years{2019b}\emph{Workshop on Ubiquitous Music} (2020).\\
\years{2019a--2021}\emph{Audio Mostly} (2019, 2021).\\
\years{2018}\emph{ACM Special Interest Group on Computer GRAPHics and Interactive Techniques (SIGGRAPH)} (2018).\\
\years{2018}\emph{International Conference of the Learning Sciences} (2018).\\
\years{2018--2020}\emph{International Society for Music Information Retrieval Conference} (2018--2020).\\
\years{2017}\emph{ACM Creativity and Cognition} (2017).\\
\years{2017}\emph{ACM Innovation and Technology in Computer Science Education} (2017).\\
\years{2017}\emph{Co-Creation Workshop at International Conference on Computational Creativity} (2017).\\
\years{2017--2018}\emph{International Computer Music Conference -- ICMC Music} (2017, 2018).\\
\years{2016, 2020}\emph{International Conference on Live Interfaces} (2016, 2020).\\
\years{2016}\emph{ISSTA International Festival and Conference on Sound in the Arts, Science and Technology} (2016).\\
\years{2016--2020}\emph{Web Audio Conference} (2016--2021).\\
\years{2015--2020}\emph{ACM Special Interest Group on Computer-Human Interaction} (2015--2021).\\
\years{2013c}\emph{IEEE Interactive Tabletops and Surfaces} (2013).\\
\years{2012--2018}\emph{ACM Designing Interactive Systems} (2012, 2016, 2018).\\
\years{2012--2018}\emph{ACM Tangible, Embedded and Embodied Interaction} (2012--2018).\\
\years{2011--2018}\emph{New Interfaces for Musical Expression} (2011--2018).

\subsection*{Journal Reviewer}
\noindent

\years{2019b}\emph{Computer Music Journal}. MIT Press Journals (2019--2020).\\
\years{2019a}\emph{Arts}. MDPI.\\
\years{2018d}\emph{PLOS One}. Public Library of Science.\\
\years{2018c}\emph{British Journal of Educational Technology}. Wiley.\\
\years{2018b}\emph{Journal of New Music Research}. ScholarOne Manuscripts.\\
\years{2018a}\emph{Transactions on Computing Education}. ACM.\\
\years{2017}\emph{Journal of Audio Engineering Society}. Audio Engineering Society.\\
\years{2016b}\emph{Interacting with Computers}. Oxford Journals.\\
\years{2016a}\emph{Qualitative Research}. Sage Publications.\\
\years{2015}\emph{International Journal of Human-Computer Studies}. Elsevier.

\subsection*{Book Reviewer}
\noindent

\years{2020--2021}\emph{Routledge}.

%\subsection*{Panelist}
%\noindent
%\years{2018e}(November 21, 2018). \emph{Panel: Women in Music Technology around the World} with Nela Brown (FLO), Magdalena Chudy (FLO), Liz Dobson (YSWN), Ada Mathea Hoel (WoNoMute), Léa Ikkache (WiMT), Tuna Pase (FLO), Franziska Schroeder (FLO), Ariane Stolfi, (Sonora) Sonia Wilkie (FLO) and Anna Xambó (WoNoMute, WiMT). Sonic Arts Research Center, Queen's University Belfast. Belfast, Northern Ireland.\\
%\years{2018d}(November 15, 2018). \emph{Panel: Future of the Music Industries} with Joe Lyske (chair, MXX), Jesper Skibsby (panelist, WARM), Nick Breen (panelist, Reed Smith) and Anna Xambó (panelist, NTNU). Resonate Music Conference 2018. Barras Art and Design (BAAD), Glasgow, Scotland, UK.\\
%\years{2018c}(October 26, 2018). \emph{Panel Session 3: Equality, Diversity, Gender} with Thomas Hilder (chair), Jill Diana Halstead Hjørnevik (panelist), Sunniva Skjøstad Hovde (panelist), Vivian Anette Lagesen (panelist), and Anna Xambó (panelist). Knowing Music -- Musical Knowing: Cross disciplinary dialogue on epistemologies. International Music Research School 2018, NTNU. Dokkhuset, Trondheim, Norway.\\
%\years{2018b}(July 4, 2018) \emph{The Disturbing Discussion about Innovation} with Nicolas d'Alessandro (panelist), Tom Mitchell (panelist), Anna Xambó (panelist), and Matthias Strobel (moderator). Wallifornia MusicTech Hackathon. Liège, Belgium.\\
%\years{2018a}(June 6, 2018) \emph{Future, Democratization, and Globalization of NIMEs} with Onyx Ashanti (panelist), Peter Nyboer (panelist), Anna Xambó (panelist), Pamela Z (panelist) and R. Benjamin Knapp (moderator). NIME '18. Moss Arts Center: Anne and Ellen Fife Theatre. Blacksburg, VA, USA.

\subsection*{Hackathon Coach}
\noindent

\years{2022}(January 29-30, 2022) On-the-Fly Live Coding Hacklab. ZKM, Karlsruhe, Germany.\\
\years{2018}(July 4, 2018) Wallifornia MusicTech Hackathon. Liège, Belgium.

\subsection*{Jury Member}
\noindent

\years{2021}\emph{Pamela Z Award for Innovation (NIME '21)}, NYU Shanghai, Shanghai, China.\\
\years{2020}\emph{Pamela Z Award for Innovation (NIME '20)}, Royal Birmingham Conservatoire, Birmingham, UK.\\
\years{2019}\emph{Pamela Z Award for Innovation (NIME '19)}, UFRGS, Porto Alegre, Brazil.\\
\years{2018}\emph{COLLAB2018}, Institute of Electronic Music and Acoustics (IEM), University of Music and Performing Arts. Graz, Austria.\\
\years{2016}\emph{MOOG Hackathon 2016}, GTCMT, Georgia Tech. Atlanta, GA, USA.

\subsection*{Music Judge}
\noindent

\years{2018}\emph{Celebrating Women in Sound, 8 March 2018}, Goldsmiths University, London.\\
\years{2017b}\emph{National Student Electronic Music Event 2017}, Louisiana State University. Baton Rouge, LA, USA.\\
\years{2017a}\emph{EarSketch National Competition 2017}, GTCMT, Georgia Tech. Atlanta, GA, USA.

%\hrule
\subsection*{Consultancies}
\noindent

\years{08/2015--10/2015}\emph{Flux Project}, Atlanta, GA, USA.\\
Consulting on the development of interactive audio components of an art project for Flux Night 2015.\\
Collaborators: Jason Freeman (coordinator), Gerard Roma.

\subsection*{Entrepreneurship}
\noindent
\years{01/2019--05/2020}\emph{WebAudioConf.com} \\
Usability supervisor and project coordinator of a sustainable website that hosts the online proceedings of the Web Audio Conference.\\
\years{08/2018--12/2019}\emph{Women Nordic Music Technology (WoNoMute)}, NTNU, Trondheim, Norway. \\
Co-Founder and Chair of \emph{WoNoMute}, an organization at NTNU in collaboration with University of Oslo (UiO) that aims to promote and connect the work of women in music tech at local, national and international levels. This includes the curation and organization of a monthly seminar series of talks, the creation of an online archive and generation of related content to give more visibility to the work of women in music tech.\\
\years{2009--present}\emph{Carpal Tunnel}, Barcelona, Spain.\\
Co-Founder of the online music record label Carpal Tunnel.\\
\years{05/2016--12/2017}\emph{Women in Music Tech}, Atlanta, GA, USA. \\
Co-Founder and Co-Chair of \emph{Women in Music Tech}, the first student organization at GTCMT that looks into bringing more women into the program of music technology with actions on recruitment, external communication, internal communication, and creating a safe space.\\
\years{02/2004--06/2010}\emph{Nodular Soft}, Barcelona. \\
Co-Founder of a freelance studio focused on user-centric software and AV communication, development of community websites using several CMS, development of AV programs under specific needs, and usability consultancy.

\subsection*{Research Visits}

\years{06/2019}Filmuniversität Babelsberg KONRAD WOLF, Potsdam, Germany.\\
\years{07/2018}Filmuniversität Babelsberg KONRAD WOLF, Potsdam, Germany.\\
\years{05/2012}University of Strathclyde, Glasgow, Scotland, UK.\\
\years{06/2011}University of Strathclyde, Glasgow, Scotland, UK.\\
\years{04/2011--05/2011} UPF, Barcelona, Spain.\\
\years{03/2010--06/2010}The Open University, Milton Keynes, UK.

%\hrule
\section*{Skills}
%\noindent

\subsection*{Languages}
\noindent

Catalan (native or bilingual proficiency), Spanish (native or bilingual proficiency), English (full professional proficiency, TOEFL iBT: 97/120 in 2009, PhD in the UK, living abroad since 2010 (UK/USA/Norway), Norwegian (basic level, Level 2 completed in 2019), German (basic level, Level 2 completed in 1991), Italian (basic level, Level 1 completed in 1991), French (basic level, self-taught).

\subsection*{Computer Skills}
\noindent

Operating Systems: OS X, Windows and Linux desktop (Ubuntu).\\
Programming: Actionscript, Assembly (basic level), C, CSS, Java, JavaScript, jQuery, MySQL, PHP, Python, Web Audio, XML.\\
Scientific Apps: MATLAB, Octave, R, SPSS.\\
Version Control Systems: CVS, Git, Subversion.\\
Music \& Audio Apps: DAWs (Ableton, Cubase, Live, Logic Pro, Reaper), Max/MSP, PureData, SuperCollider, wave editors (Audacity, SoundForge, WaveEditor), audio measurement \& test (Lindos).\\
Text \& Multimedia Analysis Apps: ELAN, MAXQDA, NVivo, VCode.\\
Other Apps: Graphics and multimedia authoring apps (AfterEffects, Blender, Dreamweaver, Final Cut Pro, Flash, Freehand, Illustrator, InDesign, Photoshop, Premiere, Processing, Combustion, 3DMax), LaTeX, MS Office suite. CMS (Drupal, WordPress). Static site generators (Jekyll, 11ty). Code editors (Atom, Visual Studio Code).\\
Online Learning Platforms: Blackboard, Canvas, Teams.\\
Hardware: Arduino, Bela, Lindos, Raspberry Pi.\\
Circuit simulators: Fritzing, Tinkercad, OrCAD.

%\vspace{1cm}
\vfill{}
%\hrulefill

\begin{center}
{\scriptsize  Anna Xambó •\- Curriculum Vitae •\- Last updated: \today\- •\- %original: Last updated: \today\- •\- 
% ---- PLEASE LEAVE THIS BACKLINK FOR ATTRIBUTION AS PER CC-LICENSE
Typeset in \href{http://nitens.org/taraborelli/cvtex}{
%\fontspec{Times New Roman}
\XeTeX }\\
% ---- FILL IN THE FULL URL TO YOUR CV HERE
\href{http://github.com/axambo/CV}{http://github.com/axambo/CV}}
\end{center}

\end{document}