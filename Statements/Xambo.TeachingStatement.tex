%------------------------------------
% Dario Taraborelli
% Typesetting your academic CV in LaTeX
%
% URL: http://nitens.org/taraborelli/cvtex
% DISCLAIMER: This template is provided for free and without any guarantee 
% that it will correctly compile on your system if you have a non-standard  
% configuration.
% Some rights reserved: http://creativecommons.org/licenses/by-sa/3.0/
%------------------------------------

%!TEX TS-program = xelatex
%!TEX encoding = UTF-8 Unicode

\documentclass[10pt, a4paper]{article}
\usepackage{fontspec} 

% DOCUMENT LAYOUT
\usepackage{geometry} 
\geometry{a4paper, textwidth=5.5in, textheight=8.5in, marginparsep=7pt, marginparwidth=.6in}
\setlength\parindent{0in}

% FONTS
\usepackage[usenames,dvipsnames]{xcolor}
\usepackage{xunicode}
\usepackage{xltxtra}
\defaultfontfeatures{Mapping=tex-text}
%\setromanfont [Ligatures={Common}, Numbers={OldStyle}, Variant=01]{Linux Libertine O}
%\setmonofont[Scale=0.8]{Monaco}
%%% modified by Karol Kozioł for ShareLaTeX use
\setmainfont[
  Ligatures={Common}, Numbers={OldStyle}, Variant=01,
  BoldFont=LinLibertine_RB.otf,
  ItalicFont=LinLibertine_RI.otf,
  BoldItalicFont=LinLibertine_RBI.otf
]{LinLibertine_R.otf}
\setmonofont[Scale=0.8]{DejaVuSansMono.ttf}

% ---- CUSTOM COMMANDS
\chardef\&="E050
\newcommand{\html}[1]{\href{#1}{\scriptsize\textsc{[html]}}}
\newcommand{\pdf}[1]{\href{#1}{\scriptsize\textsc{[pdf]}}}
\newcommand{\doi}[1]{\href{#1}{\scriptsize\textsc{[doi]}}}
% ---- MARGIN YEARS
\usepackage{marginnote}
\newcommand{\amper{}}{\chardef\amper="E0BD }
\newcommand{\years}[1]{\marginnote{\scriptsize #1}}
\renewcommand*{\raggedleftmarginnote}{}
\setlength{\marginparsep}{7pt}
\reversemarginpar

% HEADINGS
\usepackage{sectsty} 
\usepackage[normalem]{ulem} 
\sectionfont{\mdseries\upshape\Large}
\subsectionfont{\mdseries\scshape\normalsize} 
\subsubsectionfont{\mdseries\upshape\large} 

% PDF SETUP
% ---- FILL IN HERE THE DOC TITLE AND AUTHOR
\usepackage[%dvipdfm, 
bookmarks, colorlinks, breaklinks, 
% ---- FILL IN HERE THE TITLE AND AUTHOR
	pdftitle={Anna Xambó - vita},
	pdfauthor={Anna Xambó},
	pdfproducer={http://nitens.org/taraborelli/cvtex}
]{hyperref}  
\hypersetup{linkcolor=blue,citecolor=blue,filecolor=black,urlcolor=MidnightBlue} 

\pagenumbering{gobble} % switch off page numbering

%HEADER & FOOTER
\pagenumbering{arabic}
\usepackage{lastpage}
\usepackage{fancyhdr}
\pagestyle{fancy}
\renewcommand{\headrulewidth}{0pt}
\lfoot{Anna Xambó, PhD}
\cfoot{Teaching Statement}%removes pagination at the center of the footer
\rfoot{\thepage\ of \pageref{LastPage}}

% DOCUMENT
\begin{document}
{\LARGE Teaching statement}\\[0.2cm]
Anna Xambó, PhD\\
\href{http://annaxambo.me}{annaxambo.me}

\section*{My Teaching Experience and Style}

I have six years of teaching experience (1999--2005), including (1) graduate and undergraduate courses of audio, video, motion graphics, and crossmedia in two Spanish universities (Universitat Politècnica de Catalunya and Universitat de Vic) and an art institute of animation and multimedia (Media Art Institute Fak d’Art); (2) professional courses of audio and multimedia software in an IT training company (Crea Formación); and (3) preschool \& primary school courses of crossmedia in three schools (Escola Magòria, Escola Costa i Llobera, and Escola Glòries) for a research project, in which I was the PI. Thus, I am used to adapt and deliver teaching content to different groups of students.

My teaching style has been practice based and project studio based, driven by the notion of learning by solving real-world problems driven by personal interests. In particular, I am interested in promoting ownership, agency, and autonomy among students through both teamwork and individual work around their topics of interest. Students need to take agency of their projects and the teacher should guide them over the course of their journey. 

My teaching style is very interdisciplinary. For example, I co-designed a hands-on course on Crossmedia for undergraduates. This course promoted hands-on interdisciplinary work by mixing sound, image, animation, video, motion graphics, and Internet. In 2004, we were awarded with a research grant to adapt these contents to preschoolers and grade-schoolers.

I have experience with software-based classes that are practical in nature and taught to students working with their computers. My lecture-based classes try to give a more theoretical stance of a given topic related to the students projects, so a period of time in needs to be allocated for one-to-one or group-to-team discussions on the projects. Reflecting on my teaching style, it is moving closer to the Coursera model of including more theory in the class and let students explore the practical aspects by their own by providing a selected list of resources and exercises. However, group discussion in the class is important to solidify content knowledge. During the lectures, it is essential to make sure that the students are understanding the content with open questions, dialogues and quizzes. My teaching style is in alignment with ideas from (1) Dewey's experiential learning pedagogy about learning through experiences, and (2) Papert's constructionism learning about learning by making, and learning to learn.

With most technology and computing-related subjects, the content can vary from one term to the next given the fast pace of the discipline. As a professor, you need to be actively updating the contents, and having a constant attitude of learning to learn. Also, it is important to transfer learning to learn to the students and concepts that are transferable, e.g., to other software environments or real-world situations.

\section*{Teaching Materials}

I always try to provide the material in digital format, typically the slides and audiovisual material, so that students can download and master the material. I usually try to give a basic set of compulsory exercises for all levels i.e. beginners and experts, and additional voluntary exercises for those students who are more advanced. I published a book, \emph{Herramientas de Diseño Digital} (Anaya-Multimedia, 2004). This book includes the course materials that I designed and used during my teaching at the Media Art Institute Fak d'Art. This material stems from 4 years of teaching (1999-2003). I still think that publishing a book is helpful for consolidating and sharing teaching materials beyond the students from the institution, and would like to continue this path with my future teaching.

\section*{Mentoring and Guest Lectures}

As a Postdoctoral Fellow in Georgia Tech, I have been mentoring graduate students and giving guest talks with their projects. My approach is close to communities of practice, in which conversations influence the evolvement of the students' projects, but also senior students help younger students. It is rewarding to see the impact of these conversations in the quality and excellence of the final projects.

In particular, during the academic year 2015--2016 I co-advised Marc Huet and Travis Gasque (master's students in Digital Media, School of Literature, Media, and Communication) and Anna Weisling (PhD student in Digital Media, School of LMC) for their graduate design project TuneTable (see our TEI '17 publication for further details). This work has been part of Brian Magerko’s Digital Media studio course at Georgia Tech, which has informed a successful and competitive National Science Foundation (NSF) funded grant Advancing Informal STEM Learning Grant. During the second semester of the academic year 2015--2016, I also co-advised Scott Wise (graduate student in Computer Science, School of Computer Science) for a project of automatic algorithmic composition during Spring 2016. From September 2015 to May 2017, I have been co-advisor of Pratik Shah (master student in Human-Centered Computing, School of Interactive Computing) with the research and design on adding collaborative features to EarSketch, an online platform for learning code by making music. This work has been part of the design and development of the NSF-funded project EarSketch, led by Jason Freeman. From this work we have published at ICLI '16 (see Peer-Reviewed Conference Papers) and we have submitted a second conference paper and are preparing a journal article.

\section*{My Teaching Areas of Interest}

My teaching areas of interest include: human-computer interaction; interaction design; tangible, physical and social computing; arts and social sciences research methods applied to music computing; creative programming; computer music algorithms and practices; and real-time interactive systems for music performance; among others. Preparing a new class is always exciting and challenging. Also, making more visible the potential of social applications from the use of technologies (e.g., music technology) is important to attract more female students into the field.

\section*{Teaching Music Technology Related Courses}

Teaching music technology related courses should be based on the experience of own music practice. However, it can be a biased situation. The concepts on music technology can be (and should be) illustrated with the combination of own examples and others' complementary examples. In order to overcome a biased situation, the principles need to be taught separated from  personal taste and a particular aesthetics. The evaluation criteria needs to be clear for both students and teacher, so that musical aesthetics should not interfere in grading. Promoting critical listening in the discussion is essential to develop excellent students in music technology related subjects. 

%\vspace{1cm}
\vfill{}
%\hrulefill

\begin{center}
{\scriptsize  Anna Xambó •\- Curriculum Vitae •\- Last updated: \today\- •\- %original: Last updated: \today\- •\- 
% ---- PLEASE LEAVE THIS BACKLINK FOR ATTRIBUTION AS PER CC-LICENSE
Typeset in \href{http://nitens.org/taraborelli/cvtex}{
%\fontspec{Times New Roman}
\XeTeX }\\
% ---- FILL IN THE FULL URL TO YOUR CV HERE
\href{https://github.com/axambo/CV/tree/master/Statements}{https://github.com/axambo/CV/tree/master/Statements}}
\end{center}

\end{document}