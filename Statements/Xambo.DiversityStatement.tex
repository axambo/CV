%------------------------------------
% Dario Taraborelli
% Typesetting your academic CV in LaTeX
%
% URL: http://nitens.org/taraborelli/cvtex
% DISCLAIMER: This template is provided for free and without any guarantee 
% that it will correctly compile on your system if you have a non-standard  
% configuration.
% Some rights reserved: http://creativecommons.org/licenses/by-sa/3.0/
%------------------------------------

%!TEX TS-program = xelatex
%!TEX encoding = UTF-8 Unicode

\documentclass[10pt, a4paper]{article}
\usepackage{fontspec} 

% DOCUMENT LAYOUT
\usepackage{geometry} 
\geometry{a4paper, textwidth=5.5in, textheight=8.5in, marginparsep=7pt, marginparwidth=.6in}
\setlength\parindent{0in}

% FONTS
\usepackage[usenames,dvipsnames]{xcolor}
\usepackage{xunicode}
\usepackage{xltxtra}
\defaultfontfeatures{Mapping=tex-text}
%\setromanfont [Ligatures={Common}, Numbers={OldStyle}, Variant=01]{Linux Libertine O}
%\setmonofont[Scale=0.8]{Monaco}
%%% modified by Karol Kozioł for ShareLaTeX use
\setmainfont[
  Ligatures={Common}, Numbers={OldStyle}, Variant=01,
  BoldFont=LinLibertine_RB.otf,
  ItalicFont=LinLibertine_RI.otf,
  BoldItalicFont=LinLibertine_RBI.otf
]{LinLibertine_R.otf}
\setmonofont[Scale=0.8]{DejaVuSansMono.ttf}

% ---- CUSTOM COMMANDS
\chardef\&="E050
\newcommand{\html}[1]{\href{#1}{\scriptsize\textsc{[html]}}}
\newcommand{\pdf}[1]{\href{#1}{\scriptsize\textsc{[pdf]}}}
\newcommand{\doi}[1]{\href{#1}{\scriptsize\textsc{[doi]}}}
% ---- MARGIN YEARS
\usepackage{marginnote}
\newcommand{\amper{}}{\chardef\amper="E0BD }
\newcommand{\years}[1]{\marginnote{\scriptsize #1}}
\renewcommand*{\raggedleftmarginnote}{}
\setlength{\marginparsep}{7pt}
\reversemarginpar

% HEADINGS
\usepackage{sectsty} 
\usepackage[normalem]{ulem} 
\sectionfont{\mdseries\upshape\Large}
\subsectionfont{\mdseries\scshape\normalsize} 
\subsubsectionfont{\mdseries\upshape\large} 

% PDF SETUP
% ---- FILL IN HERE THE DOC TITLE AND AUTHOR
\usepackage[%dvipdfm, 
bookmarks, colorlinks, breaklinks, 
% ---- FILL IN HERE THE TITLE AND AUTHOR
	pdftitle={Anna Xambó - vita},
	pdfauthor={Anna Xambó},
	pdfproducer={http://nitens.org/taraborelli/cvtex}
]{hyperref}  
\hypersetup{linkcolor=blue,citecolor=blue,filecolor=black,urlcolor=MidnightBlue} 

\pagenumbering{gobble} % switch off page numbering

%HEADER & FOOTER
\pagenumbering{arabic}
\usepackage{lastpage}
\usepackage{fancyhdr}
\pagestyle{fancy}
\renewcommand{\headrulewidth}{0pt}
\lfoot{Anna Xambó, PhD}
\cfoot{Diversity Statement}%removes pagination at the center of the footer
\rfoot{\thepage\ of \pageref{LastPage}}

% DOCUMENT
\begin{document}
{\LARGE Diversity Statement}\\[0.2cm]
Anna Xambó, PhD\\
\href{http://annaxambo.me}{annaxambo.me}

\section*{Why Diversity Matters}

A diverse environment is important for understanding and respecting other points of view and facilitating rich collaborations. In the anthropology field it is said: \emph{understanding "the other" helps to understand oneself}. In the field of music technology, as it also happens in other STEM fields, there is a well-known problem of lack of diversity. As a woman in music technology, I am particularly concerned about the lack of women in my field, and thus the lack of role models. In particular, my actions are committed to diversity and equity in higher education.

\section*{We Are Agents of Change}

We can be agents of change. As part of the \href{http://webaudio.gatech.edu/}{WAC 2016} committee, we included a travel grant program that looked into increasing diversity in the WAC conference hosted at Georgia Tech, and made sure that there was gender parity in \href{http://webaudio.gatech.edu/keynotes}{keynote speakers}. Even though all our efforts, the number of female attendees was low. What was the problem?\\

I took it as a personal research and conducted a series of informal interviews with women in the field of music technology at different stages of their life: an undergrad student, two grad students, an early career lecturer, a high school teacher, and a music technology's program coordinator. In April 2016, I was invited as a \href{http://wiswos.bitbucket.org/index.html\%3Fp=435.html}{keynote speaker in conversation with Liz Dobson} at \emph{Women in Sound Women on Sound on Educating Girls into Sound}, held in Lancaster, UK. We discussed the findings. We continued the discussion at the Georgia Tech Center for Music Technology (GTCMT) department. Then, we decided to create \href{http://www.gtcmt.gatech.edu/}{Women in Music Tech} to make sure that we sustainably bring more women into the program.

\section*{Women in Music Tech}

\href{http://www.gtcmt.gatech.edu/womeninmusictech}{Women in Music Tech} is an exciting adventure. This organization has been possible with the collaboration of everyone from the GTCMT department: students, staff, and faculty members. We are also in conversations with a number of organizations and individuals from Georgia Tech that are collaborating, supporting, and helping us to define our agenda. We have programmed outreach events, launched a bi-monthly newsletter, and are making sure that there is a safe and respectful space for everybody. Our next milestone is to create a sustainable organization during 2017--2018. With only one year of actions, the graduate level of applications has improved but is still not ideal (female participation 21.1\% (2015--16), 16.3\% (2016--17), 25.4\% (2017--18)).

\section*{What Else Can We Do?}

As agents of change, we can improve our environment establishing constant dialogue with the department about the organization's actions and achievements. Furthermore, attending events related to diversity, such as the \href{http://www.diversity.gatech.edu/diversitysymposium}{Diversity Symposium 2016} at Georgia Tech, is important to network, join forces and share current issues and future strategies. 

%\vspace{1cm}
\vfill{}
%\hrulefill

\begin{center}
{\scriptsize  Anna Xambó •\- Diversity Statement •\- Last updated: \today\- •\- %original: Last updated: \today\- •\- 
% ---- PLEASE LEAVE THIS BACKLINK FOR ATTRIBUTION AS PER CC-LICENSE
Typeset in \href{http://nitens.org/taraborelli/cvtex}{
%\fontspec{Times New Roman}
\XeTeX }\\
% ---- FILL IN THE FULL URL TO YOUR CV HERE
\href{https://github.com/axambo/CV/tree/master/Statements}{https://github.com/axambo/CV/tree/master/Statements}}
\end{center}

\end{document}